% Acronyms
\nomenclature[aAMIP]{AMIP}{Atmospheric Model Intercomparison Project}
\nomenclature[aBOG]{BOG}{Byrne \& O'Gorman (Radiation Scheme)}
\nomenclature[aCALIPSO]{CALIPSO}{Cloud-Aerosol Lidar and Infrared Pathfinder Satellite Observations}
\nomenclature[aCOSP]{COSP}{CFMIP Observational Simulator Package}
\nomenclature[aGOCCP]{GOCCP}{GCM-Oriented CALIPSO Cloud Observations Product}
\nomenclature[aCMIP]{CMIP5/6}{The Fifth / Sixth Coupled Model Intercomparison Project}
\nomenclature[aCERES]{CERES}{Clouds and Earth's Radiant Energy System}
\nomenclature[aCFMIP]{CFMIP}{Cloud Feedback Model Intercomparison Project}
\nomenclature[aCRE]{CRE (CRF)}{Cloud Radiative Effect (Forcing)}
\nomenclature[aCWP]{CWP}{Cloud Water Path}
\nomenclature[aEIS]{EIS}{Estimated Inversion Strength}
\nomenclature[aELF]{ELF}{Estimated Low-level Cloud Fraction}
\nomenclature[aECMWF]{ECMWF}{European Centre for Medium-Range Weather Forecasts}
\nomenclature[aERA5]{ERA5}{ECMWF Reanalysis version 5}
\nomenclature[aGCM]{GCM}{General Circulation Model}
\nomenclature[aGFDL]{GFDL}{Geophysical Fluid Dynamics Laboratory}
\nomenclature[aIPCC]{IPCC}{Intergovernmental Panel on Climate Change}
\nomenclature[aIPCCR]{IPCC AR5}{The Fifth Assessment Report of the IPCC}
\nomenclature[aISCCP]{ISCCP}{International Satellite Cloud Climatology Project}
\nomenclature[aITCZ]{ITCZ}{Intertropical Convergence Zone}
\nomenclature[aLTS]{LTS}{Low Tropospheric Stability}
\nomenclature[aLCL]{LCL}{Lifting Condensation Level}
\nomenclature[aOLR]{OLR}{Outgoing Longwave Radiation}
\nomenclature[aPDF]{PDF}{Probability Density Function}
\nomenclature[aPPE]{PPE}{Perturbed Parameter (or Physics) Ensemble}
\nomenclature[aRH]{RH}{Relative Humidity}
\nomenclature[aRRTM]{RRTM}{Rapid Radiative Transfer Model}
\nomenclature[aSOCRATES]{SOCRATES}{Suite Of Community RAdiative Transfer codes based on Edwards \& Slingo}
\nomenclature[aSPOOKIE]{SPOOKIE}{Selected Process On/Off Klima Intercomparison Experiment}
\nomenclature[aSST]{SST}{Sea Surface Temperature}
\nomenclature[aSW]{SW}{Shortwave}
\nomenclature[aLW]{LW}{Longwave}
\nomenclature[aTOA]{TOA}{Top of the Atmosphere}

% Matrix, vector, scalar
%\nomenclature[ba]{$\bm X$}{A matrix.}
%\nomenclature[ba]{$z$}{A scalar.}
%\nomenclature[bb]{$\bm x_j$}{A column vector; the $j$th column of matrix $\bm X$; sometimes written as $\bm x_{:,j}$. There is one exception to this rule: the precision matrix $\bm B$ in chapter \ref{chap:2} retains its capitalisation to avoid confusion with other variables.}

% Roman letters
\nomenclature[ra]{$a$ }{Earth radius in \chapref{ch:polaramplification}; linear coefficient in \chapref{ch:simple_cld_scheme}}
\nomenclature[rC]{$C$ }{Cloud fraction}
\nomenclature[rCw]{$C_w$ }{Specific heat capacity of ocean water (J kg$^{-1}$ K$^{-1}$)}
\nomenclature[rD]{$D$ }{Depth of mixed-layer (m)}
\nomenclature[rF]{$F$, $F_{2\times}$ }{Radiative forcing; radiative forcing from a doubling of CO$_2$ (W m$^{-2}$)}
\nomenclature[rFs]{$F_s$, $F_Q$ }{Surface flux or Q-Flux (W m$^{-2}$)}
\nomenclature[rR]{$R$}{Radiative flux at top of the atmosphere (W m$^{-2}$)}
\nomenclature[rK]{$K$}{Radiative kernel (W m$^{-2}$ K$^{-1}$)}
\nomenclature[rTs]{$T_s$, $T_0$}{Surface temperature or reference temperature (K)}
\nomenclature[rT]{$T$}{Temperature (K)}
\nomenclature[rS0]{$S_0$}{Solar constant  (W m$^{-2}$)}
\nomenclature[rg]{$g$}{Gravitational acceleration ($9.8$ m s$^{-2}$)}
\nomenclature[rq]{$q$}{Specific humidity (kg kg$^{-1}$)}
\nomenclature[rqs]{$q_s$}{Saturation mixing ratio or specific humidity (kg kg$^{-1}$)}
\nomenclature[rqsurf]{$q_0$, $q_{700}$}{Specific humidity at surface or 700 hPa (kg kg$^{-1}$)}
\nomenclature[rqt]{$q_t$, $q_c$}{Total water or cloud water mixing ratio (kg kg$^{-1}$)}
\nomenclature[rH]{$H$}{Relative humidity in \chapref{ch:simple_cld_scheme}; heat transport in \chapref{ch:polaramplification}}
\nomenclature[rHc]{$H_c$}{Critical relative humidity}
\nomenclature[res]{$e_s$}{Saturated water vapor pressure}
\nomenclature[res0]{$e_{s0}$}{Reference saturation vapor pressure (6.1078 hPa) at 273.16 K}
\nomenclature[rcp]{$c_p$}{Dry air specific heat capacity at constant pressure (1006 J kg$^{-1}$ K$^{-1}$)} % https://glossary.ametsoc.org/wiki/Specific_heat_capacity
\nomenclature[rLv]{$L_v$}{Latent heat of vaporization ($2.47\times 10^{6}$ J kg$^{-1}$)}
\nomenclature[rRa]{$R_a$}{Dry air gas constant ($287.04$ J kg$^{-1}$ K$^{-1}$)}
\nomenclature[rRv]{$R_v$}{Gas constant for water vapor ($461.50$ J kg$^{-1}$ K$^{-1}$)}
\nomenclature[rGm]{$\Gamma_m$}{Moist-adiabatic potential temperature gradient (K m$^{-1}$)}
\nomenclature[rp]{$p$}{Pressure (hPa)}
\nomenclature[rp0]{$p_0$, $p_s$}{Reference pressure or sea level pressure (hPa)}
\nomenclature[rz]{$z_{700}$, $z_{LCL}$, $z_{inv}$}{Heights of 700 hPa, lifting condensation level (LCL) or inversion layer (m)}
\nomenclature[rre]{$r_e$}{Effective radius of cloud condensate ($\mu$m)}
\nomenclature[rwl]{$w_l$}{In-cloud liquid water mixing ratio (g kg$^{-1}$)}
\nomenclature[rfl]{$f_l$}{Liquid cloud fraction}


% Greek letters
\nomenclature[ga]{$\alpha$}{Surface albedo}
\nomenclature[gd]{$\delta$, $\Delta$}{Change of a variable}
\nomenclature[gl]{$\lambda$}{Climate feedback parameter}
\nomenclature[gw]{$\omega$}{Vertical pressure velocity (Pa s$^{-1}$ or hPa day$^{-1}$)}
\nomenclature[gw500]{$\omega_{500}$, $\omega_{700}$}{Vertical pressure velocity at 500 hPa or 700 hPa (Pa s$^{-1}$ or hPa day$^{-1}$)}
\nomenclature[gr]{$\rho$, $\rho_w$}{Density; density of water (kg m$^{-3}$)}
\nomenclature[ge]{$\epsilon$}{Surface emissivity}
\nomenclature[gs]{$\sigma$}{Stefan--Boltzmann constant ($5.67\times 10^{-8}$ W m$^{-2}$ K$^{-4}$)}
\nomenclature[gh]{$\theta$}{Potential temperature (K)}
\nomenclature[ghs]{$\theta_s$, $\theta_{700}$}{Potential temperature at surface or 700 hPa (K)}
\nomenclature[gt]{$\tau$}{Optical depth}
\nomenclature[gv]{$\varphi$}{Latitude}

% Greek alphabet ordering
% a alpha
% b beta
% c gamma
% d delta
% e epsilon
% f zeta
% g eta
% h theta
% i iota
% k kappa
% l lambda
% m mu
% n nu
% o xi
% p omicron
% q pi
% r rho
% s sigma
% t tau
% u upsilon
% v phi
% x chi
% y psi
% z omega
