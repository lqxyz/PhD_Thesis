% thesisPreamble.tex

% \pagenumbering{roman}
% \clearpage
% %\pagenumbering{gobble}
\maketitle

%Abstract goes here.  Approx 300 words.
\begin{abstract}
\addcontentsline{toc}{chapter}{Abstract}
%This is part of abstract.

The global mean surface air temperature change in response to global warming, namely climate sensitivity, plays a central role in climate change studies, and the estimates of climate sensitivity depend critically on the climate feedbacks, the processes that can either amplify or dampen the responses of climate system to the external perturbations. The goal of this thesis is to understand the climate feedbacks through the idealized climate models.

The first part explores the roles of climate feedbacks played in polar amplification of surface temperature change. By running the idealized aquaplanet simulations with a hierarchy of radiation schemes (without sea ice and clouds), and by decomposing the total surface temperature responses into different components through the radiative kernel method, we find the poleward heat transport, the lapse rate and Planck feedbacks contribute to the amplified surface temperature changes in polar region, while the water vapor feedback prefers the tropical temperature change.

The second part investigates the underlying causes of cloud feedback uncertainty with a simple cloud scheme. The scheme diagnoses the cloud fraction from relative humidity and other variables such as inversion strength, and its optical properties such as effective radius and cloud water content are prescribed as simple functions of temperature. The simulations show this scheme can capture the basic feature of cloud climatology. Through a series of perturbed parameter ensemble global warming simulations, part of the inter-model spread of cloud feedbacks among general circulation models can be reproduced, and we find the low cloud amount feedback, especially the one over the low-latitude subsidence regions, is the largest contributor to the net cloud feedback uncertainty. The cloud controlling factor analysis suggests that the sea surface temperature (SST) and estimate inversion strength (EIS) have opposite effects on marine low cloud amounts, but their responses to SST rather than EIS seem to bring much larger uncertainty. Finally, we find the equilibrium climate sensitivity and cloud feedback over tropical subsidence regimes show a robust linear relationship, implying a possible constraint for climate sensitivity.

\end{abstract}

%% Declaration
\makedeclaration

%% Acknowledgements
%\makededication
\newpage
\chapter*{Acknowledgements}
\addcontentsline{toc}{chapter}{Acknowledgements}
% generous guidance, support and endless patience
First and foremost, I would like to express my sincerest thanks to my supervisors, Prof. Mat Collins and Prof. Geoff Vallis, for their generous guidance, support and endless patience throughout my PhD. The discussions with them (mostly) on Fridays have greatly shaped my research. They always encourage me to think big pictures of scientific problems, which I will benefit in the rest of my life. I would also like to thank Dr. Penelope Maher and Dr. Stephen Thomson for coding the initial version of the simple cloud scheme, which forms the basis of the scheme developed in Chapter 4. My gratitude is also due to Neil Lewis for developing the column version of Isca model, which makes the tests of parameterization scheme much easier.
% They always encourage me to think big pictures. 
% The invaluable discussions 
% When I was in down times, they always encourage me
% They always encourage me to think big pictures of the scientific problems, which makes sure I can return on the right track.
% They are always there when I need help and they always encourage me to think big pictures of the scientific problems. 

I am thankful to Prof. James Screen for working as an assessor for my annual report and for his advice and suggestions during every mini-viva. Many thanks to all the people who have helped me along the way in Exeter, especially those in my office on Laver level 9 and the members of the `Isca group'. I would also like to extend my thanks to the Exeter Climate System (XCS) group for organizing so many excellent seminars, although all the discussions have to be online during the COVID-19 pandemic.

%I would also like to extend my thanks to the Exeter Climate System (XCS) group and Geophysical and Astrophysical Fluid Dynamics (GAFD) group for organizing so many excellent seminars.

I would like to acknowledge the joint PhD scholarship from University of Exeter and China Scholarship Council (NO. 201706210070), as I could not finish my study in the UK without their support.

Finally, I am incredibly grateful to my family and friends for their continuous care and support over the past four years. Every week's chat with my parents and brother have given me lots of support, especially during the COVID-19 lockdown period. I also owe thanks to my housemates and friends for their support and help.

% They always respect my choice,
%My father Shugang and mother Hezhen, who are always there for me when I need any support, advice and help. My elder twin brother Yang, who has obtained his PhD degree two years ago, sets a good example for me to follow. %, and who is always the first person I think of when I want to share some good or bad news.

% Chinese visiting scholars (share office on the same level at Laver): Peng Wang (HKCU), Lei Wang (IAP), Runnan Zhang (Fudan, Associate Researcher), Yimian Ma (IAP), Peiqiang Xu (IAP, postdoc), Jun Ying (2nd Ocean Institute, Associate Prof)
% Xiaomin Huang (Nanjing Agriculture Univ, NAU), Ping Liao (Jiangxi Agriculture Univ)
%
% Roommates: Xiaomin Huang, Xiao Zhou (Tongji), Yue Yuan, Yuefeng Shao (Southeast Univ), Kai Yu (NAU)

%--------------------------------------------------------------
% List of contents
%--------------------------------------------------------------
\newpage
\pagestyle{fancy}
%\setcounter{tocdepth}{4}
% Reset linkcolor to black
% https://tex.stackexchange.com/questions/59462/change-ref-color-of-on-the-list-of-figures
\bgroup
\hypersetup{linkcolor=black}
% https://tex.stackexchange.com/questions/61033/setting-toc-depth-not-working
\setcounter{tocdepth}{2}
\tableofcontents
\egroup

%---------------------------------------------------------------
% List of tables, illustrations etc
%---------------------------------------------------------------
\newpage
\addcontentsline{toc}{chapter}{List of Tables}
%\chapter*{List of tables}
\bgroup
\hypersetup{linkcolor=black}
\listoftables
\egroup

\newpage
\addcontentsline{toc}{chapter}{List of Figures}
\bgroup
\hypersetup{linkcolor=black}
\listoffigures
\egroup
% Short caption in list of figures
% https://tex.stackexchange.com/questions/11579/short-captions-for-figures-in-listoffigures
% \caption[Short version for LoF]{Long version to appear next to the figure}

%--------------------------------------------------------------
% List of accompanying material (if any)
%--------------------------------------------------------------

%--------------------------------------------------------------
% Definitions (terms specific to the work); abbreviations
%--------------------------------------------------------------
