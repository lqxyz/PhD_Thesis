\chapter{Code and data availability}

\section{The simple cloud scheme}
%The simple cloud code can be accessed at \url{https://doi.org/10.5281/zenodo.4382536}, and the updates can be found at \url{https://github.com/lqxyz/Isca/tree/simple_clouds}. It will be mergred with Isca master repository (\url{https://www.github.com/ExeClim/Isca}) in the future. 

\subsection{Introduction}
The simple cloud scheme diagnoses cloud fraction based on relative humidity (RH) and specifies the in-cloud water mixing ratio and effective radius of the cloud condensate as function of temperature. It has been implemented and tested under Isca framework \citep{Vallis2018} and can be ported to other climate models if needed.

\subsection{Code structure}

The simple cloud scheme code can be accessed at \url{https://doi.org/10.5281/zenodo.4382536}, and the updates can be found at \url{https://github.com/lqxyz/Isca/tree/simple_clouds}. It will be merged with the Isca master repository (\url{https://github.com/ExeClim/Isca}) in the future. Specifically, they are in \url{src/atmos\_param/cloud\_simple} directory and are called by the file \normalfont{\url{src/atmos\_spectral/driver/solo/idealized\_moist\_phys.F90}}.

The major files in \textit{src/atmos\_param/cloud\_simple} directory include:
\begin{itemize}
    \item \textit{cloud\_simple.F90} \\
    The main module of the SimCloud scheme, which specifies the in-cloud water mixing ratio and effective radius of cloud condensate, and calls the following modules to diagnose cloud fraction.
    
    \item \textit{large\_scale\_cloud.F90}\\
    The module that diagnoses large-scale clouds based on RH. In this module, several different schemes are provided, such as \textit{linear} and \textit{\cite{Sundqvist1989}} schemes, which can be set through \textit{large\_scale\_cloud\_nml} namelist by specifying the method name (\textit{cf\_diag\_formula\_name}).
    
    \item \textit{marine\_strat\_cloud.F90}\\
    This module diagnoses the marine stratus clouds based on low-level cloud proxy \textit{ELF} (estimated low-level cloud fraction) from \cite{Park2019}.

    \item \textit{cloud\_cover\_diags.F90}\\
    This module diagnoses the 2D cloud cover based on different overlap assumptions, including \textit{`maximum-random'}, \textit{`maximum'} and \textit{`random'}.
\end{itemize}

\section{Data and scripts}

\subsubsection{\chapref{ch:polaramplification}}

The derived radiative kernels for three radiation schemes of Isca are availabel at \url{https://doi.org/10.5281/zenodo.4282681}, namely two gray radiation schems, Frierson \citep{Frierson2006} and  Byrne and O'Gorman \citep[BOG;][]{Byrne2013}, and a full radiation scheme, the multiband correlated-$k$ Rapid Radiative Transfer Model \citep[RRTM;][]{Clough2005}. The scripts to calculate the offline radiative kernels can be found at \url{https://github.com/lqxyz/Isca_kernels} (last access: 31 July 2021). The input basic state data sets are available at \url{https://doi.org/10.5281/zenodo.4071837}, and they are generated from Isca simulations (T42, 25 vertical levels) without sea ice and clouds.

\subsubsection{\chapref{ch:simple_cld_scheme}}

The Isca model outputs are available on Zenodo: \url{https://doi.org/10.5281/zenodo.4573610}, generated from the AMIP-type fixed SST simulations with realistic continents, SOCRATES radiation scheme and the simple cloud scheme with different setups. An archive of the scripts used to process data and generate figures/tables is available at \url{https://doi.org/10.5281/zenodo.4597263} and the updates can be found at \url{https://github.com/lqxyz/cloud_scheme_manuscript_figs} (last access: 31 July 2021).

\subsubsection{\chapref{ch:cld_fbk}}

The Isca perturbed parameter ensemble (PPE) simulation outputs under 1$\times$CO$_2$ and 4$\times$CO$_2$ scenarios are available on Zenodo: \url{https://doi.org/10.5281/zenodo.5150241} and \url{https://doi.org/10.5281/zenodo.5188175}. The scripts for processing data and plotting figures can be found at \url{https://github.com/lqxyz/cloud_feedback_from_Isca_PPE} (last access: 3 August 2021).
