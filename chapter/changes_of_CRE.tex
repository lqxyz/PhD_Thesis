\chapter{Changes of Cloud Radiative Effect to Global Warming}

\section{Introduction}
In \chapref{ch:eval_cld_scheme} we have evaluated the climatology simulated by the simple cloud scheme when implemented in Isca, and we will investigate how cloud radiative effect responses to climate change in this chapter. 

\section{Data and model setup}
Similar to the setups in \secref{sec:eval_cld_exp_setup}, here the SST has a uniform 4K warming.


\section{Cloud simulator}
As introduced in \secref{sec:Zelinka_method}, one possible way to calculate the cloud feedback is to use the cloud radiative kernel \citep{Zelinka2012computing1,Zelinka2012computing2}, in which the joint-histogram between cloud top pressure and optical depth from ISCCP cloud simulator is needed. One can simply estimate the cloud feedback by the change of cloud radiative effect between control and perturbed experiments, but it also includes the masking effect of climatological cloudiness on non-cloud feedbacks \citep{Soden2004}. In addition, combining the histogram outputs from ISCCP cloud simulator with cloud radiative kernel, one can decompose the cloud feedback into  different altitude and different components \citep{Zelinka2012computing2,Zelinka2016insights}, which is a helpful tool for us to understand the causes of spread in multi-model spread of cloud feedback. As mentioned in \secref{ch:eval_cld_scheme}, the cloud simulator has not been implemented in Isca, and we could not employ the cloud radiative kernel method to calculate and decompose cloud feedback without it. Thus in this section, we try to implement the cloud simulator in Isca first in order for a refined look into cloud feedback.

In this study, the Cloud Feedback Model Intercomparison Project (CFMIP) Observational Simulator Package (COSP) version 2 \citep{Swales2018}, which can be accessed at \url{https://github.com/CFMIP/COSPv2.0}, is implemented in Isca. COSP was originally developed as a satellite simulator package whose aim is to produce virtual satellite observations from atmospheric model fields for a better comparison of model output with observations \citep{Bodas2011}. This approach is needed because the satellite retrievals generally do not directly correspond to the numerical model fields due to the mismatch between their definitions of certain fields. COSP accounts for the limited view of the satellite instrument by calculating radiative transfer through the atmosphere, i.e. attenuation by hydrometeors and air molecules and backscattering \citep{Kuma2020}. Note that multiple instrument simulators, such as MODIS, CALIPSO, CloudSat and ISCCP, have been incorporated in COSP, and it is flexible for users to decide which one to use based on their research purposes. 

Specifically, several modules have been written in Isca to call COSP, in which the outputs from simple cloud scheme and SOCRATES radiation schemes, such as cloud fraction, effective radius, cloud water content and cloud optical depth, are provided through the interfaces. However, as the cloud scheme is simple and there is not microphysics scheme in Isca, we could not provide some properties about convective clouds and cloud condensate such as ice and graupel. Although this may bring some problems, the outputs from ISCCP simulator are relatively reasonable.


\section{Summary and discussion}