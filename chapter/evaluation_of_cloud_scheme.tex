\chapter{Evaluation of the Cloud Scheme}

\section{Introduction}

In this chapter, the simulated climatology from the simple cloud scheme described in \chapref{chp:simple_cld_scheme} is presented here. \secref{sec:exp_setup_and_dataset} introduces the experiment setup and data sets used in this chapter. The evaluation will focus on features of cloud fraction, cloud water path and cloud radiative effects, including their vertical profile (if any), spatial pattern, zonal mean structure and seasonal cycles.


\section{Experiment setup and data sets}
\label{sec:exp_setup_and_dataset}

The simple cloud scheme was implemented into Isca \citep{Vallis2018} to examine its performance. Isca is an open-source framework for the construction of general circulation of atmospheres, which is built around a dynamical core from the Geophysical Fluid Dynamics Laboratory (GFDL) and physical parameterizations from \citet{Frierson2006} and \citet{Frierson2007}. Isca provides various options for users to set up experiments for their own interests, which include dry and moist models, various convection and radiation schemes, a variety of land/sea configurations and different parameters for other planetary atmospheres.

\section{Simulated cloud fraction}

\section{Simulated cloud water path}

\section{Simulated cloud radiative effect}

\section{Comparison with CMIP5 models}

\section{Discussion and conclusions}

