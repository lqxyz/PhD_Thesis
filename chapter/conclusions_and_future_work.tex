\chapter{Conclusions and Future Work}
\label{ch:conclusion}

\section{Conclusions}
In this thesis, my goal is to understand the climate feedbacks with simple climate models and the following problems are addressed: 
\begin{enumerate}
    \item What roles do the climate feedback processes play in the polar amplification of surface temperature change in idealized aquaplanet simulations?
    \item Could we build a relatively simple cloud scheme to study the cloud feedback related problems?
    \item If so, could we use it to explore the underlying causes of the intermodel spread of cloud feedback in GCMs? 
\end{enumerate}

To answer the first question, a series of Isca \citep{Vallis2018} aquaplanet simulations (without sea ice and clouds) with different surface albedos and with a hierarchy of radiation schemes were performed (see \chapref{ch:polaramplification}). In the simulations, the climate feedbacks are quantified through the radiative kernel method, derived from the offline calculation with the radiation codes \citep{Liu2020kernel}. When the total temperature response is decomposed into different components, we find that the increase in poleward heat transport, the lapse rate and Planck feedbacks contribute to the polar amplification most, while the water vapor feedback prefers the tropical temperature change.

%The reason for us to build a simple cloud scheme is that 
%  the cloud feedback has the largest uncertainty in current CMIP models

The second and the third problems try to understand the cloud feedback with a simple cloud scheme. The reason to use a simple scheme rather than the existing ones is that previous studies have suggest that the spread does not decrease even if the convection schemes are off \citep{Webb2015} and the cloud scheme itself might play a role in it \citep[e.g.,][]{Qu2014,Geoffroy2017}. As the idealized climate model involved in this study is Isca, which has no cloud scheme at the beginning, the first step is to construct a simple cloud scheme for it. As introduced in \chapref{ch:simple_cld_scheme}, the scheme \citep{Liu2021simcloud} is inspired by SPOOKIE II project, which intends to explore the role of cloud scheme in the inter-model spread of cloud feedback. In the scheme, the large-scale clouds are diagnosed from relative humidity, and the marine low stratus clouds, typically found off the west coast of continents over subtropical oceans, are determined largely as a function of inversion strength. A ``freeze-dry" adjustment based on a simple function of specific humidity is also available to reduce an excessive cloud bias in polar regions. The cloud optical related properties, such as the effective radius of cloud droplet and cloud liquid water content, are specified as simple functions of temperature. All of these features are user-configurable. The AMIP fixed-SST simulations show the scheme can capture the spatial pattern, zonal mean structure and seasonal cycle of climatologies of cloud radiative effect, and hence could be a useful tool for clouds related study. 

To answer the third question, a series of perturbed parameter ensemble (PPE) simulations under control (CO$_2$ level is 300 ppm) and global warming (quadruple CO$_2$) situations are performed in Isca. In the simulations, the perturbed parameters are limited to cloud scheme and the runs are performed with realistic continents and with prescribed Q-flux. 

With this scheme, we are investigating whether we could reproduce part of inter-model uncertainty of cloud feedback by perturbing its key parameters.The preliminary results show the low cloud amount feedback, especially in the tropical and subtropical regions, is the largest contributor to the uncertainty, and the possible reasons for this uncertainty are being analysed through the cloud controlling factor analysis.

\section{Future work}

For the climate feedback part?

For cloud scheme:

Problems of RH schemes?

Diagnostic --> inconsistency between cloud fraction of cloud water?

Aerosol? missing feature of negative optical depth feedback...

Coupling with convection scheme? Cloud life-time?

===============================================

For the cloud feedback:

The perturbation in PPE is so limited: 1. only with the cloud scheme, 2. the perturbation is not enough for the single parameter? increase and/or decrease in a range so as to quantify the roles of the parameters? 

The coupling between clouds and large-scale circulation? Its role in determining the cloud feedback spread?

Pattern effect? difference between Q-flux and AMIP runs?

For ECS?

The role of clouds in polar amplification?


%\section{Conclusions}