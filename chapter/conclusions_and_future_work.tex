\chapter{Conclusions and Future Work}
\label{ch:conclusion}

\section{Conclusions}

The global mean surface temperature change in response to increased greenhouse gas (i.e., climate sensitivity) is one of the key problems in current climate studies \citep[e.g.,][]{Bony2006,Stocker2013,Sherwood2020}, and the estimates of climate sensitivity depend on the climate feedbacks \citep{Bony2006,Soden2006}. The aim of this thesis is to understand the climate feedbacks with simple climate models and the following problems are addressed: 
\begin{enumerate}
    \item What roles do the climate feedback processes play in the polar amplification of surface temperature change in idealized aquaplanet simulations?
    \item Could we build a cloud scheme for idealized general circulation models (GCMs) that is simple enough but could grasp the key features of cloud fields?
    \item If so, could the scheme be used to investigate the underlying causes of the intermodel spread of cloud feedback in climate models? 
\end{enumerate}

To answer the first question, a series of Isca \citep{Vallis2018} aquaplanet simulations (without sea ice and clouds) with different surface albedos and with a hierarchy of radiation schemes were performed (see \chapref{ch:polaramplification}). In the simulations, the climate feedbacks are quantified through the radiative kernel method, derived from the offline calculation with the radiation codes \citep{Liu2020kernel}. When the total temperature response is decomposed into different components, we find that the increase in poleward heat transport, the lapse rate and Planck feedbacks contribute to the polar amplification most, while the water vapor feedback prefers the tropical temperature change.

%The reason for us to build a simple cloud scheme is that 
%  the cloud feedback has the largest uncertainty in current CMIP models

The second and the third problems were to understand the cloud feedback with a simple cloud scheme. The reason to use a simple scheme rather than the existing ones is that previous studies have suggest that the spread does not decrease even if the convection schemes are off \citep{Webb2015} and the cloud scheme itself might play a role in it \citep[e.g.,][]{Qu2014,Geoffroy2017}. Therefore, a simple scheme that interacts directly with radiation scheme only might be helpful. As the idealized climate model involved in this study is Isca, which has no cloud scheme at the beginning, the first step is to construct a simple cloud scheme for it. As introduced in \chapref{ch:simple_cld_scheme}, the scheme \citep{Liu2021simcloud} is inspired by the phase II of Selected Process On/Off Klima Intercomparison Experiment (SPOOKIE II) project, which intends to explore the role of cloud scheme in the intermodel spread of cloud feedback. In the scheme, the large-scale clouds are diagnosed from relative humidity, and the marine low stratus clouds, typically found off the west coast of continents over subtropical oceans, are determined largely as a function of inversion strength. A ``freeze-dry" adjustment based on a simple function of specific humidity is also available to reduce an excessive cloud bias in polar regions. The cloud optical related properties, such as the effective radius of cloud droplet and cloud liquid water content, are specified as simple functions of temperature. All of these features are user-configurable. The Atmospheric Model Intercomparion Project (AMIP) fixed sea surface temperature (SST) simulations show the scheme can capture the spatial pattern, zonal mean structure and seasonal cycle of climatologies of cloud radiative effect, and hence could be a useful tool for clouds related study. 

To answer the third question, a series of perturbed parameter ensemble (PPE) simulations under control (CO$_2$ level is 300 ppm) and global warming (quadruple CO$_2$) situations are performed in Isca (see \chapref{ch:cld_fbk}). In the PPE simulations, the perturbed parameters are limited to cloud scheme and the runs are performed with realistic continents and with prescribed Q-flux. We find the simple cloud scheme can capture two robust positive cloud feedbacks, low cloud amount feedback and high cloud altitude feedback, but fails to grasp the strong negative optical depth feedback in midlatitudes, as the mixed-phase clouds not explicitly represented in the simple cloud scheme. The PPE simulations could reproduce part of intermodel uncertainty of cloud feedback in GCMs by perturbing its key parameters, and the low cloud amount feedback, especially in the subsidence regime of tropical and subtropical oceans, is the largest contributor to the net cloud uncertainty. The cloud controlling factor analysis suggests that the SST and estimated inversion strength (EIS) have opposite effects on marine low cloud amounts, but their responses to SST rather than EIS seem to bring larger uncertainty. As for the climate sensitivity, we find the perturbation related to marine low clouds produces the largest cloud feedback and equilibrium climate sensitivity (ECS), and the range of ECS perhaps can be constrained by the tropical cloud feedback over subsidence regimes, as they show a robust linear relationship in Isca PPE and sixth phase of Coupled Model Intercomparison Project (CMIP6) models.

\section{Future work}

In this thesis, we have investigated the 
roles of climate feedbacks in climate system with a series of idealized simulations, but in fact there are still lots of work can be done in the future.

In \chapref{ch:polaramplification}, we have explored the roles of different climate feedbacks in amplified change of surface temperature in polar regions. However, due to the limitation of our setup, the roles of sea ice and clouds in polar amplification have not been studied. As suggested by \cite{Screen2010}, the retreat of sea ice may play a central role in Arctic temperature amplification. As there is no sea ice model in Isca, perhaps in the future we can set up a simple sea ice model to explore the roles of sea ice in polar amplification. Alternatively, we can use current aquaplanet setup, but just modify the albedo in polar regions to mimic the diminishing of sea ice, so as to investigate the role of sea ice in a simple way. Note that the simple cloud scheme has not been constructed when the simulations described in \chapref{ch:polaramplification} were run, so the roles of clouds are ignored. But as the simple cloud scheme is ready now, we could rerun the same simulations with the simple cloud scheme to examine the role of clouds in the same problem, and to see whether other feedbacks would change when clouds are present.

% as assessed by \cite{Kim2018}, the cloud radiative effects are negative in high latitudes due to an increase of low­level cloud amount and positive in low latitudes due to a decrease of overall cloud amount under global warming

The work presented in \chapref{ch:simple_cld_scheme} and \chapref{ch:cld_fbk} is based on the simple cloud scheme described in \cite{Liu2021simcloud}. At first, we should keep in mind that the scheme is a diagnostic scheme, and the cloud fraction and cloud water are diagnosed from different variables in the scheme, which could lead to the inconsistency between cloud fraction and cloud condensate \citep[e.g.,][]{Gregory2002,Tompkins2005}. Also, cloud fraction and relative humidity might show opposing changes at some locations under global warming in RH scheme \citep{Ming2018}. Therefore, in the future, perhaps we could build a statistical scheme or even a prognostic scheme in Isca, so as to improve the simulations of cloud fields. As we discussed in \secref{sec:spatial_pattern_cld_fbk}, the negative optical depth feedback is too weak in current simple cloud scheme, due to the poor representation of mixed-phase clouds and lack of mircophysical processes, which makes the global mean cloud feedback larger than the ensemble mean of CMIP5/6 models. In future studies, we consider introducing the microphysical processes in the simple cloud scheme, so as to better simulate the transition between ice and liquid cloud and to improve the simulation of cloud lifetime and its optical depth feedback. 

For the Isca PPE simulations in \chapref{ch:cld_fbk}, currently only the parameter within the cloud scheme are perturbed, and only one parameter is perturbed each time, so the parameter space is narrow compared to the intermodel differences. In addition, the PPE results might be biased due to the perturbations are within a small scale. To solve these potential problems, it is worthy to perturb more parameters to generate a wider parameter space in the future. For example, we can perturb the parameters from other parameterization schemes, or can perturb several parameters at the same time. In this way we can explore the possible ranges of ECS.

Moreover, it is noteworthy that not only cloud scheme itself, other factors such as SST warming pattern \citep[e.g.,][]{Zhou2016,Dong2019attributing,Dong2020intermodel} or the coupling between clouds and circulation \citep[e.g.,][]{Bony2004,Vial2013} can also have impacts on cloud feedbacks, but these issues are not explored in this study. In the future, we could design a series experiments to test these problems with the simple cloud scheme. For example, we could run the simulations with different warming patterns to examine the influences on climate feedbacks. And we can investigate the cloud-circulation coupling with the cloud-locking method \citep[e.g.,][]{Voigt2020review}. All these efforts will help us better understand cloud feedbacks in climate system.

%Also, the cloud scheme are directly coupled with convection scheme

%For the climate feedback part?
%For cloud scheme:
%Problems of RH schemes?
%Diagnostic --> inconsistency between cloud fraction of cloud water?
%Aerosol? missing feature of negative optical depth feedback...
%Coupling with convection scheme? Cloud life-time?
%
% ===============================================
%
% For the cloud feedback:
% The perturbation in PPE is so limited: 1. only with the cloud scheme, 2. the perturbation is not enough for the single parameter? increase and/or decrease in a range so as to quantify the roles of the parameters? 

% The coupling between clouds and large-scale circulation? Its role in determining the cloud feedback spread?
% Pattern effect? difference between Q-flux and AMIP runs?
%
% For ECS?
% The role of clouds in polar amplification?
