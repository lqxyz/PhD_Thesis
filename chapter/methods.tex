\chapter{Data and Methods}
\label{ch:methods}

In this thesis, several different observational and reanalysis data sets, as well as model simulations, are used. For example, the reanalysis data sets are employed in \chapref{ch:simple_cld_scheme} to derive the linear formula between cloud fraction and relative humidity, and to evaluate the relationship between low cloud amount and its possible proxies. The satellite products relating to clouds and radiation fields are used in \chapref{ch:simple_cld_scheme} to evaluate the performance of the simple cloud scheme. To make it clear, these data sets are briefly summarized in this chapter. One major topic of this thesis is the cloud feedback, and the main tool used to simulate it is the idealized climate model, Isca \citep{Vallis2018}. Therefore, the features of this climate model and the advantages of using the idealized models are discussed here. However, how to calculate or estimate cloud feedback in climate models is not so evident, and the direct estimate from the change of cloud radiative effect might be impacted by the cloud masking effect. Therefore, it is necessary to discuss the pros and cons of the several different methods to estimate cloud feedback in this chapter. %Also, based on these discussion, it would be much easier for Isca to calculate the cloud feedback and decompose it into different components if the cloud simulator is implemented there, so the topic about this is also presented in this chapter. 

The chapter is organized as follows: The satellite data sets related to radiation and cloud fields are introduced in \secref{sec:obs_reanalysis_dataset}. The idealized general circulation model employed throughout this thesis is described briefly in \secref{sec:isca_intro}. In \secref{sec:cosp}, how the cloud observation simulator package is implemented in Isca is documented. \secref{sec:method_cloud_fbk} summarizes the possible methods to calculate the cloud feedback. The low cloud amount proxies used in \chapref{ch:simple_cld_scheme} are listed in \secref{sec:method_low_cld_amount_proxy} for reference.

\section{Observational and reanalysis data sets}
\label{sec:obs_reanalysis_dataset}

%\subsection{Satellite datasets}
\subsection{Clouds and radiation data sets}
Satellites have played an important role in observing the Earth over recent decades, and these satellite retrievals are essential data sets for us to understand the weather and climate on Earth. The observation of clouds and radiation at the top of the atmosphere (TOA) depends particularly on the satellites. In general, the satellite products can be categorized into different groups according to their processing levels ranging from Level 0 to Level 4 \citep{Parkinson2006earth}. Specifically, Level 0 is the raw and unprocessed instrument data. Level 1 is annotated with ancillary information (1A) and processed to sensor units (1B). Level 2 is derived geophysical variables, Level 3 is mapped to a space-time grid, and Level 4 are modeled outputs or variables derived from multiple measurements. It is usually easier for users to use these satellite retrievals when they have higher processing levels. In this thesis, most satellite data sets used are at Level 3 or 4, and only one at Level 2, as to be introduced below.

% (see \url{https://earthdata.nasa.gov/collaborate/open-data-services-and-software/data-information-policy/data-levels}, last accessed: April 28, 2021)
% https://nsidc.org/the-drift/2013/08/is-it-1b-2-or-3-definitions-of-data-processing-levels/

% \begin{itemize}
%     \item \textbf{Level 0}:	Reconstructed, unprocessed instrument and payload data at full resolution, with the communications artifacts removed.
%     \item \textbf{Level 1A}: The Level 0 data with time-reference and with ancillary information
%     \item \textbf{Level 1B}: Same as Level 1A data, but have been processed to sensor units (not all instruments have Level 1B source data)
%     \item \textbf{Level 2}:	Derived geophysical variables at the same resolution and location as Level 1 source data.
%     \item \textbf{Level 3}:	Variables mapped on uniform space-time grid scales, usually with some completeness and consistency.
%     \item \textbf{Level 4}:	Model output or variables derived from multiple measurements.
% \end{itemize}

\subsubsection{Clouds and the Earth’s Radiant Energy System  Top-of-Atmosphere (TOA) data product}
% Energy Balanced and Filled (EBAF)
\index{Clouds and the Earth’s Radiant Energy System}

The Clouds and the Earth’s Radiant Energy System (CERES) project (\url{https://ceres.larc.nasa.gov}) \index{CERES} provides satellite-based observations of Earth's radiation budget and clouds \citep{Wielicki1996clouds}. It uses measurements from CERES instruments flying on several satellites (TRMM\footnote{TRMM: Tropical Rainfall Measuring Mission, a research satellite in operation from 1997 to 2015}, Terra, Aqua, S-NPP\footnote{S-NPP: Suomi National Polar-orbiting Partnership}, NOAA-20\footnote{NOAA-20: National Oceanic and Atmospheric Administration-20}), along with measurements from higher-resolution imagers on polar orbiting and geostationary satellites and other input data source, to produce a comprehensive set of Earth's radiation budget data products for weather, climate and applied science research. The goal of CERES is to produce a long-term, integrated global climate data record for detecting the possible changes in the Earth’s radiation budget from the surface to the top of the atmosphere, so as to improve our understanding of how Earth’s radiation budget varies in time and space and of the role that clouds and other atmospheric properties play.

The main CERES product used in this thesis is the Energy Balanced and Filled (EBAF) product \citep[CERES-EBAF hereafter;][]{Loeb2009}, especially the latest version Ed4.1 \citep{Loeb2018}. The CERES-EBAF product involves the CERES and Moderate Resolution Imaging Spectroradiometer (MODIS) instruments flying on the Terra (descending sun-synchronous orbit with an equator crossing time of 10:30 A.M. local time) and Aqua (ascending sun-synchronous orbit with an equator crossing time of 1:30 P.M. local time) as well as geostationary imagers that provide hourly diurnal information between 60$^\circ$S--60$^\circ$N \citep{Loeb2018}. The EBAF data set is designed to solve the existing imbalance issue in the average global net radiation at the TOA from the CERES satellite observations. It uses an objective constraint algorithm to adjust shortwave and longwave TOA fluxes within their ranges of uncertainty to remove the inconsistency between average global net TOA flux and heat storage in the Earth-atmosphere system. The EBAF product consists of monthly mean shortwave, longwave, and net TOA all-sky and clear-sky radiative fluxes over 1$^\circ \times$1$^\circ$ latitude–longitude regions, as well as the MODIS-based cloud properties such as cloud amount, optical depth, effective pressure, and temperature at cloud top. In \chapref{ch:simple_cld_scheme}, this CERES-EBAF Ed4.1 product is used to evaluate the cloud radiative effect simulated from the simple cloud scheme.

\subsubsection{International Satellite Cloud Climatology Project (ISCCP) -- H Series}
\index{International Satellite Cloud Climatology Project}

The International Satellite Cloud Climatology Project (ISCCP) \index{ISCCP} began in early 1980s as part of the World Climate Research Program (WCRP) \citep{Schiffer1983,Rossow1991}, and is probably the longest-running international satellite-based global environmental data project. Its goal is to collect weather satellite radiance measurements and to analyze them to infer the global distribution of clouds, their properties, and their diurnal, seasonal and interannual variations. This project has accumulated cloud observation for almost four decades and has therefore been widely used in cloud related research.

Several series of the ISCCP products have been publised previously. However, the widely used ISCCP D-series product \citep{Rossow1999advances} has not been updated since December 2009. Recently, a new H-series product was published \citep{Young2018}, spanning from 1982 to 2015 (see \url{https://isccp.giss.nasa.gov/products/onlineData.html}, last accessed: April 28, 2021). Comparing to the previous ISCCP D-series product, the H-series has improved the low-level cloud sensitivity over snow and ice in polar regions. More importantly, based on ISCCP’s legacy and in light of the technological advancements that include active spaceborne sensors (e.g., Cloud–Aerosol Lidar and Infrared Pathfinder Satellite Observations and CloudSat) and cloud data sets that rely on newer passive imagers with higher spectral, spatial, radiometric, and temporal resolutions, the H-series has exploited to produce cloud products with much higher resolutions \citep{Young2018}.

In this study, the high-resolution global monthly (HGM) product (at Level 3) with 1$^\circ \times$1$^\circ$ resolution is used for analysis (e.g., in \chapref{ch:introduction}). Of course, gridded hourly high-resolution output (HGH) is also available. In ISCCP H-series product, it has provided many cloud related variables, such as (but not limited to) cloud amount (low, middle, high and total), cloud-top temperature, cloud-top pressure, cloud optical thickness, cloud water path, cloud phase and cloud type.

\subsubsection{CloudSat Radar-Only Cloud Water Content (2B-CWC-RO) product}

The CloudSat Radar-Only Cloud Water Content (2B-CWC-RO) product contains retrievals of cloud liquid and ice water content, effective radius of the ice and liquid particles, number concentration, and the size distribution width parameter using radar and auxiliary temperature data \citep{Austin2009}. The latest release P1\_05 is used in this study. Retrievals are performed separately for the liquid and ice phases assuming liquid only and ice only, then the two sets combined in a simple way to obtain a composite profile that is consistent with the input measurements. The partition of ice and liquid is based purely on temperature, and the solutions of ice and liquid are scaled linearly with temperature to obtain a smooth transition from all ice at −20$^\circ$C to all liquid at 0$^\circ$C. The retrieval uses a temperature dependent a priori of lognormal (a modified gamma distribution) size distribution parameters, which are fitted from in situ particle size spectra.  
% Refer to: https://agupubs.onlinelibrary.wiley.com/doi/full/10.1029/2020EA001147

In this study, we choose to use the CloudSat 2B-CWC-RO product for cloud water path rather than the CERES-EBAF data set, because previous study has shown that the CloudSat 2B-CWC-RO product can better represent cloud liquid and ice water path over high latitudes than CERES-EBAF data set, owing to its explicit determination of cloud phase \citep{Lenaerts2017}. But we should notice that the CloudSat 2B-CWC-RO product also has some potential problems. For example, \cite{Barker2008} has shown the temperature-partitioned profiles could not capture the mixed phased cloud structure accurately. In this thesis, the CloudSat 2B-CWC-RO product is used in \chapref{ch:simple_cld_scheme}.

%The cloud water path is from the CloudSat \index{CloudSat} 2B-CWC-RO Release P1\_R05 data product \citep{Austin2009} from 2012 to 2016, which can better represent cloud liquid and ice water path over high latitudes than CERES-EBAF data set, owing to its explicit determination of cloud phase \citep{Lenaerts2017}. The original CloudSat cloud water path data set has some missing data for certain dates, and is not in the T42 resolution we are going to use. To get the annual mean product we desired, the original data set is processed as follows: For the area each grid point covers, the total times that the satellite has visited over the 5 years and the sum of the cloud water path that has been retrieved over same period are recorded, from which the annual mean cloud water path is derived. The scripts to do this can be accessed at \url{https://github.com/lqxyz/cloudsat_cloud_water_path}.

\subsubsection{GCM-Oriented CALIPSO Cloud Product}

The General Circulation Model (GCM)-Oriented Cloud-Aerosol Lidar and Infrared Pathfinder Satellite Observation (CALIPSO) Cloud Product (GOCCP) \citep[CALIPSO-GOCCP hereafter;][]{Chepfer2010} is designed to evaluate the cloudiness simulated by GCMs. It contains observational cloud diagnostics fully consistent with the ones simulated by the GCM plus the lidar simulator, that is they have similar spatial resolution, the same criteria used for cloud detection and the same statistical cloud diagnostics. In doing so, the Cloud-Aerosol Lidar with Orthogonal Polarization (CALIOP) Level 1 data (raw data but annotated with ancillary information) are processed following the steps in a lidar simulator to diagnose the model cloud cover that CALIPSO would observe from space if the satellite were flying above an atmosphere similar to that predicted by the GCM. 

The CALIPSO-GOCCP products include the vertical distribution of cloud fraction, horizontal distribution of low, middle, high, and total cloud fractions, instantaneous lidar scattering ratio profiles, and scattering ratio histograms as a function of height. Hence, the cloud cover outputs from GCM can be compared directly with this product when the cloud simulator was not implemented. More importantly, the CALIPSO-GOCCP data can be used to evaluate the cloud fraction profiles simulated from climate models. In this thesis, the CALIPSO-GOCCP is used in \chapref{ch:simple_cld_scheme}.

%\subsection{Reanalysis data sets}
\subsection{Other climatology variables}

% \begin{figure}[ht]
%     \centering
%     \includegraphics[width=0.9\linewidth]{{figs/methods/reanalysis_ecmwf}.pdf}
%     \caption[A schematic of the reanalysis process]{A schematic of the reanalysis process, taken from the European Centre for Medium-Range Weather Forecasting (ECMWF) website (see \url{https://www.ecmwf.int/en/about/media-centre/focus/2020/fact-sheet-reanalysis}, last accessed: April 30, 2021).}
%     \label{fig:reanalysis}
% \end{figure}

In this thesis, the basic climate state variables such as temperature, specific humidity, relative humidity and dynamical fields such as zonal and meridional winds, pressure velocity (e.g., $\omega_{500}$) and mean sea level pressure are taken from the reanalysis data sets. The reanalysis is produced by combining the past short-range weather forecasts with observations through data assimilation, which can provide multi-variate, globally complete, consistent record of the atmosphere. The reanalysis has some advantages over the observation. For example, the observations are not always evenly distributed, and even in the satellite era, observations alone cannot provide a complete picture of the state of the Earth system across the globe at a given point in time. Also, the reanalysis is usually the best choice currently  when there is no complete observation of a certain variable, or when data sets which are consistent with each other are required.

In this thesis, the European Centre for Medium-Range Weather Forecasting (ECMWF) Interim Reanalysis \citep[ERA-Interim;][]{Dee2011} and the latest ERA5 \citep{era5} reanalysis are used. The ERA-Interim is produced by the data assimilation system based on Cy31r2 of the Integrated Forecasting System (IFS), which includes a 4-dimensional variational analysis (4D-Var) with a 12-hour analysis window. The spatial resolution of the data set is approximately 80 km (T255 spectral) on 60 levels in the vertical from the surface up to 0.1 hPa \citep{Dee2011}. This ERA-Interim is available from 1 January 1979 to 31 August 2019, and now is replaced by the ERA5. ERA5 is the latest reanalysis product from ECMWF, which is based on the IFS Cy41r2 and covers the period from 1950 to present. The atmospheric variables from ERA5 has a regular latitude-longitude grid (0.25$^\circ \times$0.25$^\circ$) with 37 pressure levels from 1000 to 1 hPa \citep{Hersbach2020era5}.

In \chapref{ch:simple_cld_scheme}, the hourly outputs from ERA5 reanalysis are used to derive the relationship between the cloud fraction and relative humidity. The monthly outputs from ERA-Interim are used to assess the relationship between low cloud amount and its proxy. In \chapref{ch:simple_cld_scheme}, the monthly pressure velocity from ERA-Interim reanalysis is used to distinguish the dynamical regimes over tropical region.

\section{Idealized climate model - Isca}
\label{sec:isca_intro}

\subsection{Overview}

In this section, the idealized climate model, Isca\footnote{Isca is the ancient name of the city Exeter in Roman times and is also an ancient word for running water; see more at \url{https://execlim.github.io/IscaWebsite/} (last access: May 1, 2021)} \citep{Vallis2018}\index{Isca}, is introduced, which is the model employed for all the simulations in this thesis. Isca\index{Isca} is an open-source framework for the idealized modeling of global circulation of atmospheres developed at the University of Exeter \citep{Vallis2018}, which uses the dynamical core and the Flexible Modeling System (FMS, see \url{https://www.gfdl.noaa.gov/fms/}) software infrastructure from the Geophysical Fluid Dynamics Laboratory (GFDL), and the physical parameterizations from \citet{Frierson2006} and \citet{Frierson2007}. Isca is coded in Fortran and the scripts to configure and run the experiments are written in Python. The code is public available at \url{https://github.com/ExeClim/Isca}.

Isca provides various options for users to set up experiments for their own interests, so that they can explore the models with different levels of complexity under the same framework. These options include the dry and moist models, various convection and radiation schemes, a variety of land/sea configurations and different parameters for other planetary atmospheres.

\subsection{Several parameterization schemes}

\subsubsection{Convection and large-scale condensation}

Isca includes various convective parameterization schemes, specifically a simple dry scheme following \cite{Walker2006eddy}, the original Bettts--Miller scheme \citep[a covective relaxation scheme;][]{Betts1986,BettsMiller1986}, a simplified Betts--Miller scheme \citep[SBM;][]{Frierson2007}, and the relaxed Arakawa--Schubert (RAS) scheme \citep[a mass-flux based scheme;][]{Moorthi1992relaxed}.

The SBM scheme \index{Simplified Betts--Miller scheme} used in this thesis is a moist adjustment convection scheme developed by \cite{Frierson2007} and the modifications described in \cite{OGorman2008hydrological}. In the SBM scheme, the convection acts to relax the temperature and humidity to the post-convective reference profiles. Based on the first-guessed reference profiles, the next step is to determine whether there will be deep, shallow, or no convection based on some criteria, then to relax towards the reference profiles. Also, the scheme will do some corrections if the enthalpy is not conserved during the relaxation. In this convection scheme, deep convection and non-precipitating shallow convection are both included, and the convective precipitation can be derived from the deep convection. In addition, the precipitation can be partitioned into rain and snow simply by the temperature, but this is not enabled in this study.

The large-scale condensation scheme \index{Large-scale condensation scheme} used in Isca is from \cite{Frierson2006}, which is accomplished by adjusting the humidity in super-saturated regions to the saturated values immediately, with temperatures adjusted to reflect this condensation. The precipitation falls out immediately, but is re-evaporated below. The precipitation diagnosed from this scheme is regarded as large-scale precipitation.

\subsubsection{Radiation scheme}

For the radiation scheme in Isca, the choices include two gray radiation schemes (gray means that radiation in different wavelength is treated equally), which we call Frierson \citep{Frierson2006} and Byrne \& O'Gorman \citep[BOG;][]{Byrne2013}; an intermediate scheme with two infrared bands and one solar band, similar to \cite{Geen2016}; and two full radiation schemes, the multiband correlated-$k$ Rapid Radiative Transfer Model \citep[RRTM;][]{Clough2005} and the SOCRATES (Suite Of Community RAdiative Transfer codes based on Edwards and Slingo) radiation scheme \citep{Edwards1996, Manners2015}. 

In the clouds related simulations in this thesis, Isca uses the SOCRATES radiation scheme,\index{SOCRATES radiation scheme} as implemented in \cite{Thomson2019}. It is the radiation scheme used by UK Met Office for Earth and planetary science \citep{Manners2015}. Compared to RRTM radiation scheme, SOCRATES is more flexible in terms of atmospheric composition and the spectral properties of the radiation scheme (e.g. number of bands for longwave and shortwave), as these properties can be specified by the spectral files separately. More bands in spectral files have more accurate simulation results, but they will make the model run much more slowly. Currently, the default spectral files used in Isca are from the Met Office Unified Model's Global Atmosphere version 7 (GA7), which have 9 longwave bands and 6 shortwave bands \citep{Walters2019}.

% Add introduction about Gray Radiation scheme if the polar amplification is also included in the thesis

\subsubsection{Land/sea configuration}

For Earth-like simulations, Isca can use land/sea configurations with various complexities. For example, Isca can run with slab ocean without land \citep[aquaplanet setup; e.g.,][]{Geen2018regime,Geen2019processes}, or run with idealized continent shapes \citep[e.g.,][]{Pietschnig2019}, or run with realistic continents and topography \citep[e.g.,][]{Geen2018regime}. In the model, the land is treated the same with the slab ocean, except that the physical properties such as mixed layer depth, albedo, roughness length seen by the boundary-layer scheme, and moisture availability over prescribed areas are different.

In general, the prescribed mixed layer depth for land is smaller than  that for ocean, and thus the land has lower heat capacity compared to the ocean. In this thesis, the mixed layer depth of land is 1/10 of ocean's mixed layer depth, so the mixed layer depth for ocean and land are 20 m and 2 m, respectively. In addition, the albedo of the land is 1.3 times that of the ocean in setups, so that land can reflect more shortwave radiation than ocean. The moisture availability is controlled by introducing the ``evaporative resistance" parameters into the equation for surface evaporation ($E$):
\begin{equation}
    E = \rho_a C|\mathbf{v}_a|\beta(\alpha q_s - q_a),
\end{equation}
where $\rho_a$, $|\mathbf{v}_a|$ and $q_a$ are the density, wind speed, and specific humidity at the lowest model level, respectively; $C$ is the drag coefficient; and $q_s$ is the saturation specific humidity at the surface temperature \citep{Frierson2006}. Evaporative resistance parameters ($\alpha$ and $\beta$) are used to adjust the surface evaporation flux and their values can be chosen by the users. Typically, one of them might be 1 and the other is between 0 and 1, and such values will reduce evaporation from land regions. As recommended in \cite{Vallis2018}, using $\alpha=1$ and $0<\beta<1$ has the advantage of not allowing $E$ to change sign. In this study, the default values are $\alpha=1$ and $\beta=0.6$. Currently, there is no dynamical ocean in Isca and the slab ocean with prescribed mixed layer depth is adopted in the model. In this way, a closed atmospheric energy budget can be achieved, as the slab ocean responds to the incident fluxes at the sea surface, with no dynamic heat transport. In fact, the horizontal heat transport in the ocean can be achieved in Isca by prescribing the Q-flux, which is the topic of \secref{sec:Q_flux_method}.

\subsection{Boundary conditions for Isca}
\label{sec:boundary_condition_isca}

\begin{figure}[ht]
	\centering
	\includegraphics[width=0.5\linewidth]{{figs/methods/input_sst_sic_annual_mean}.png}
	\caption[Boundary conditions for Isca, including sea surface temperature, sea ice concentration and topography.]{Boundary conditions for Isca, including (a) the annual mean sea surface temperature (SST, units: K; orange) and topography over land (altitude, units: m), and (b) annual mean sea ice concentration (SIC; units: \%). The SST and SIC are derived from input4MIPs data sets \citep{Durack2018} over the period from 1979 to 2008, and the topography is derived from ERA-Interim reanalysis \citep{Dee2011}.}
	\label{fig:input_sst_ice}
\end{figure}

For the Atmospheric Model Intercomparison Project (AMIP)-type simulations \citep{Gates1999amip} in Isca, the boundary conditions such as sea surface temperature (SST) and sea ice concentration are specified by the files. In this thesis, the monthly sea surface temperature (SST) and sea ice concentration are fixed at their climatology, which are derived from the input4MIPs data sets (input data sets for Model Intercomparison Projects) \citep{Durack2018} over the period from 1979 to 2008, and the scripts to process the input data set are available at \url{https://github.com/lqxyz/input4MIPs}. The annual mean spatial patterns of the derived SST and sea ice concentration are shown in \figref{fig:input_sst_ice}. The SST over the tropical and subtropical Southeast Pacific is cooler than the West Pacific.

% Topography file
The topography and land/sea mask are from the ERA-Interim reanalysis \citep{Dee2011}. The original topography is smoothed to reduce the Gibbs ripples arising from the truncated spherical harmonic expansion (The method is provided by Greg Colyer and Ruth Geen). The final topography over land is displayed in \figref{fig:input_sst_ice}a.

\subsection{Ocean heat transport (Q-flux)}
\label{sec:Q_flux_method}

%The simulations in \chapref{ch:simple_cld_scheme} are the AMIP-type fixed-SST experiments, and the changes in atmosphere could not feedback into the SST changes. 

Another feature in Isca is that it can include ocean heat transport (the `Q-flux') \index{Q-flux|(} from a given distribution of SSTs \citep{Vallis2018}. Here we repeat how to derive the Q-flux following the method from \cite{Russell1985}. After prescribing the Q-flux in simulations, the SST can respond freely to the CO$_2$ forcing. The observed SST climatology is from the input4MIPs data set \citep{Durack2018} over the period from 1979 to 2008 (see \secref{sec:boundary_condition_isca} for details). To derive the Q-flux, the aforementioned SST climatology is prescribed in a simulation with the realistic continents, topography and a slab ocean (mixed layer depth is 20 m). The energy balance model is applied as follows:
\begin{equation}
    F_Q = C_w\rho_w D \frac{\partial T}{\partial t} - F_s,
    \label{eq:Q_flux}
\end{equation}

\begin{equation}
    F_s = \mathrm{SW}-\mathrm{LH}-\mathrm{SH}-\mathrm{LW},
    \label{eq:surface_flux}
\end{equation}
where $F_Q$ is the Q-flux to be derived, which distributes energy globally to match the prescribed SST distribution. The $C_w$ (3989.24 J kg$^{-1}$ K$^{-1}$) and $\rho_w$ (1035 kg m$^{-3}$) are the specific heat capacity and density of ocean water, respectively. $D$ is the depth of mixed layer. The rate of change of mixed layer temperature ($\frac{\partial T}{\partial t}$, in units of K s$^{-1}$) is calculated by the prescribed SST. The surface flux ($F_s$, positive downward, units Wm$^{-2}$) in \Eqref{eq:Q_flux} is calculated from the upward latent and sensible heat fluxes ($LH$ and $SH$, respectively), the net longwave radiation ($LW$, positive upward) and net shortwave radiation ($SW$, positive downward), as indicated by \Eqref{eq:surface_flux}.

\begin{figure}[ht]
	\centering
	\includegraphics[width=0.8\linewidth]{{figs/methods/Q-flux_cmp}.png}
	\caption[Comparison of spatial pattern of ocean heat transport (Q-flux).]{Annual mean Q-flux pattern (in W m$^{-2}$) derived from AMIP fixed-SST experiments with realistic continents and topography. The cloud schemes used are (a) linear, (b) linear\_FD and (c) linear\_ALL as in \tabref{tab:exps} and (d) no cloud scheme.}
	\label{fig:Q_flux_cmp}
\end{figure}

The AMIP fixed-SST experiments with linear cloud scheme described in \tabref{tab:exps} and a new fixed-SST simulation without clouds are used to derived the Q-flux, and the results are shown in \figref{fig:Q_flux_cmp}. The spatial patterns of Q-flux from different cloud schemes are similar, and have some differences from the simulation without cloud in subtropical and Southern Ocean regions. Q-flux can capture the ocean currents such as the Gulf Stream and cold tongue in the eastern tropical Pacific (\figref{fig:Q_flux_cmp}), and the positive value compensates for too little heating to the slab ocean by the surface flux from the `prescribed-SST’ run compared to the SST climatology. In this case, the Q-flux obtained from the run with the linear cloud scheme only is used in the following simulations. In addition, the Q-flux remains the same in the control and perturbed experiments, but it is noted that the SST can change freely in response to different CO$_2$ forcing. \index{Q-flux|)}

%\subsection{Experiments}

%\section{Climate feedback}
\section{CFMIP Observation Simulator Package}
\label{sec:cosp}
\index{CFMIP Observation Simulator Package}

This section records how the Cloud Feedback Model Intercomparison Project (CFMIP) Observational Simulator Package (COSP)\index{COSP} was implemented in Isca. Note that the COSP was not included in Isca simulations in \chapref{ch:simple_cld_scheme}, but as to be introduced in \secref{sec:method_cloud_fbk}, it is useful to calculate cloud feedback and to decompose the cloud feedback into different components. The outputs from the COSP are employed in \chapref{ch:cld_fbk}. 

In this study, the latest version (version 2) of COSP \citep{Swales2018} is implemented in Isca, which is publicly available at \url{https://github.com/CFMIP/COSPv2.0}. COSP was originally developed as a satellite simulator package whose aim is to produce virtual satellite observations from atmospheric model fields for a better comparison of model output with observations \citep{BodasSalcedo2011}. This approach is needed because the satellite retrievals generally do not directly correspond to the numerical model fields due to the mismatch between their definitions of certain fields. COSP accounts for the limited view of the satellite instrument by calculating radiative transfer through the atmosphere, i.e. attenuation by hydrometeors and air molecules and backscattering \citep{Kuma2020}. Note that multiple instrument simulators, such as MODIS, CALIPSO, CloudSat and ISCCP, have been incorporated in COSP, and it is flexible for users to decide which one to use based on their research purposes. Specifically, several modules have been written in Isca to call COSP, in which the outputs from simple cloud scheme and SOCRATES radiation schemes, such as cloud fraction, effective radius, cloud water content and cloud optical depth, are provided through the interfaces. However, as the cloud scheme is simple and there is no microphysics scheme in Isca, we could not provide some properties about convective clouds and cloud condensate such as ice and graupel. Although this may bring some problems, the outputs from ISCCP simulator are relatively reasonable.

\section{The calculation of cloud feedback}
\label{sec:method_cloud_fbk}
\index{Cloud feedback!Calculation method|(}

% \subsection{Introduction}
% As introduced in \chapref{ch:introduction}, the concept of climate feedback is used to characterize the response to the climate system to external radiative forcing. If we assume the climate system is at an equilibrium state, then when an external forcing $\Delta Q$ is imposed to the climate system, which can be aroused from the perturbation of CO$_2$ concentration, change in solar constant, or volcanic eruption, the climate system will response to this radiative imbalance at the top of the atmosphere (TOA) by changing the surface temperature ($T_s$). Let $R$ be the radiation budget at TOA, and the radiation imbalance ($\Delta R$) goes to zero during the climate system adjusts towards a new equilibrium. Using the idea of feedback, $\Delta R$ can be linked to the change of surface temperature ($\Delta T_s$) as follows:
% \begin{equation}
%     \Delta R = \Delta Q + \lambda \Delta T_s,
%     \label{eq:imbalance_forcing_lambda}
% \end{equation}
%in which $\lambda$ is called climate feedback parameter. Besides surface temperature, other variables such as atmospheric temperature, water vapor, albedo and cloud properties may also change, and thus might have an impact on $\lambda$ as well. In theory, the climate feedback parameter $\lambda$ can be evaluated as follows
% \begin{equation}
%     \lambda = \frac{\partial R}{\partial T_s} = \sum_x \frac{\partial R}{\partial x}\frac{\partial x}{\partial T_s}  + Re, %\text{high-order terms},
%     %+  \sum_x\sum_y \frac{\partial^2 R}{\partial x\partial y}\frac{\partial x\partial y}{\partial^2 T_s}+\dots 
%     \label{eq:lambda}
% \end{equation}
% where $x$ represent the climate variable such as temperature, water vapor, albedo and cloud properties. The high-order residual term ($Re$) is usually neglected in analysis.

As introduced in \chapref{ch:introduction}, the concept of climate feedback is used to characterize the response to the climate system to external radiative forcing. If we assume the climate system is at an equilibrium state, then when an external forcing $\Delta F$ is imposed to it, the system will response to this perturbation by a series of feedback processes. When the climate system adjusts towards a new equilibrium, the relationship among the radiation imbalance ($\Delta R$) at the top of the atmosphere (TOA), the forcing and surface temperature change  ($\Delta T_s$) can be expressed in \Eqref{eq:imbalance_forcing_lambda}, with climate feedback ($\lambda$) in \Eqref{eq:lambda}.

\subsection{Short summary of three approaches}
\label{sec:three_climate_fbk_methods}

Several different methods have been proposed to diagnose climate feedbacks in GCMs. Three main approaches are summarized below, and readers may refer to Appendix B of \cite{Bony2006} for a clear summary of the first two methods.% and refer to several pioneering papers from Soden, Held and others for a detailed description of the third method \citep[e.g.,][]{Soden2006,Soden2008,Shell2008}.

\begin{enumerate}[label={(\arabic*)}]
    \item \textbf{The PRP approach}\\
    The partial radiative perturbation (PRP) \index{Partial radiative perturbation} method \citep{Wetherald1988cloud} evaluates partial derivatives of model TOA radiation with respect to changes in model parameters by re-running the model radiation code offline. For example, to calculate the climate feedback parameter associated with $x$, the radiative response that results from the perturbation of $x$ is calculated as follows:
    %substituting one variable at a time from the perturbed climate state into the control climate.
    \begin{equation}
        \delta_x R = R(a,b,...,x') - R(a,b,...,x),
        \label{eq:delta_R_x}
    \end{equation}
    where $a$, $b$ etc are climate variables except $x$ and they are kept unchanged during calculation. $x'$ is the value of $x$ from perturbed climate state. In doing so, $\frac{\partial R}{\partial x}$ in \eqref{eq:lambda} can be obtained from this offline calculation. The feedback parameter associated with $x$ is finally computed from $\lambda_x = \frac{\partial R}{\partial x}\frac{\mathrm{d} x}{\mathrm{d} T_s}$, and $\frac{\mathrm{d} x}{\mathrm{d} T_s}$ is calculated by differencing the simulation outputs from two experiments or from different time periods. Note that a more accurate two-sided PRP method \citep{Colman1997} is also used to estimate the partial radiative responses at TOA, so \eqref{eq:delta_R_x} can be rewritten as
    \begin{equation}
        \delta_x R = \frac{1}{2} \left[R(a,b,...,x') - R(a,b,...,x) + R(a',b',...,x') - R(a',b',...,x) \right].
    \end{equation}
    The advantage of PRP method is that it can separate different climate feedbacks, including ones related to clouds. But the procedure can be computationally expensive \citep{Soden2008}, and the calculation  must be repeated for every simulation and climate model versions.
    % The parameters  from control and perturbed climate states are substituted one by one into radiation code to obtain the partial responses at TOA.
    
    \item \textbf{The CRE approach}\\
    The `cloud radiative effect' (CRE) approach \citep[or CRF in][]{Cess1990intercomparison, Cess1996cloud} decomposes the climate feedback into clear-sky and cloudy components. They do so by decompose the total TOA radiation budget $R$ as the sum clear-sky component ($R_{\text{clear}}$) and CRE ($\mathrm{CRE}=R-R_{\text{clear}}$), so \eqref{eq:imbalance_forcing_lambda} can be written as
    \begin{equation}
        \lambda = \frac{\Delta R - \Delta F}{\Delta T_s} = \underbrace{\frac{\Delta R_{\text{clear}} - \Delta F}{\Delta T_s}}_{\text{clear-sky climate feedback}} + \underbrace{\frac{\Delta\mathrm{CRE}}{\Delta T_s}}_{\text{cloud feedback}}.
        \label{eq:CRE_approach}
    \end{equation}
    It is clear that this method can not separate the clear-sky climate feedback components into temperature, water vapor and surface albedo ones in further. As we will discuss in \secref{sec:CRE_or_PRP}, the cloud feedback parameter (second term in \eqref{eq:CRE_approach}) obtained from this approach depends on changes in both clear-sky and cloudy properties, which in fact can not separate the cloud feedback completely from the clear-sky components. Nevertheless, the calculation of this method is straightforward and still valuable in GCM evaluations.
    
    \item \textbf{The radiative kernel method}\\
    The radiative kernel method \citep[e.g.,][]{Soden2006,Soden2008,Shell2008} decompose each climate feedback into two parts. The first is the `radiative kernel' ($\partial R/\partial x$), which describes the change of TOA fluxes in response to a standard change in property $x$ and depends on the radiative properties and base state of the model. The second term is the climate response of feedback variable normalized by surface temperature change ($\mathrm{d} x/\mathrm{d} T_s$). Most importantly, \cite{Soden2008} has shown the climate feedbacks calculated from three different radiative kernels are quite similar, although the radiative transfer code is different, indicating that a single kernel can be used to perform a first-order comparison of feedbacks across multiple models \citep{Shell2008}. One problem is that the kernel calculation is also computationally expensive as it requires running the offline radiative transfer code for perturbation at each model level and time step, but luckily it just needs one-time calculation and the kernel can be applied for different experiments and models. % Currently, more radiative kernels from different GCMs are available as summarized in \tabref{tab:rad_kernels}.
\end{enumerate}   
    
% \begin{table}
% 	\caption{Summary of available radiative kernels for several GCMs}
% 	% %(Part from \url{https://climate.rsmas.miami.edu/data/radiative-kernels/index.html})
% 	\vspace{0.5em}
% 	\centering
% 	\small
% 	\renewcommand{\arraystretch}{1.2}
% 	\resizebox{\textwidth}{!}{
% 	\begin{tabular}{lccll}
% 		\hline
% 		Source & TOA & Surface & Reference & Note (Data link) \\
% 		\hline
%         GFDL & $\checkmark$ & $\checkmark$ & \cite{Soden2008}  & \\
%         CESM-CAM3 & $\checkmark$ & $\checkmark$ & \cite{Shell2008} & \url{http://people.oregonstate.edu/~shellk/kernel.html}\\
%         \multirow{2}{*}{CESM-CAM5} & \multirow{2}{*}{$\checkmark$} & \multirow{2}{*}{$\checkmark$} & \cite{Pendergrass2018}  & \url{https://github.com/apendergrass/cam5-kernels}\\
%          & & & or \cite{Huang2017}  & \url{https://huanggroup.wordpress.com/publication/} \\
%         %{CESM-CAM5} &{$\checkmark$} & {$\checkmark$} & \cite{Pendergrass2018} or \cite{Huang2017}  \\
%         HadGEM3-GA7.1 & $\checkmark$ & $\checkmark$ & \cite{Smith2020} & \url{https://doi.org/10.5281/zenodo.3594673}  \\
%         \multirow{2}{*}{CloudSat} & \multirow{2}{*}{$\checkmark$} & \multirow{2}{*}{$\checkmark$} & \multirow{2}{*}{\cite{Kramer2019}} & \multirow{2}{*}{Derived from the CloudSat fluxes and } \\
%          & & & & heating rates data product.\\
% 		\hline
% 	\end{tabular}}
% 	\label{tab:rad_kernels}
% \end{table}

%These approaches summarized above are proposed for general climate feedback parameters, which of course can be used to diagnose cloud feedbacks. for the cloud feedback

As for the cloud feedback, \cite{Zelinka2012computing1,Zelinka2012computing2} proposed a new method combing the cloud radiative kernel and the ISCCP-type histogram of cloud fraction partitioned into cloud-top pressure (CTP) and optical depth ($\tau$) bins, which can quantify the cloud feedbacks from different components such as cloud amount, height and optical depth, as further discussed in \secref{sec:Zelinka_method}. %Next I try to answer the question that whether the CRE method is a proper method used to diagnose cloud feedback parameter.

\subsection{Using $\Delta$CRE to estimate cloud feedback?}
\label{sec:CRE_or_PRP}

In this section, we try to answer the question whether we could use the change of CRE between perturbed and control climate states to estimate the cloud feedback.
% In the partial radiative perturbation (PRP) method, the cloud feedback is defined as 
% \begin{equation}
%     \lambda_C = - \frac{\partial \overline{R}}{\partial C}\frac{\text{d}C}{\text{d}T_s}
% \end{equation}
According to \cite{Soden2004}, in the PRP method the change of radiation flux at TOA due to clouds ($\delta_C {R}$) is calculated as follows:
\begin{equation}
    %\delta_C \overline{R} = \overline{R}(T, C', w, \alpha_s)  - \overline{R}(T, C, w, \alpha_s),
    \delta_C R = {R}(T, C', w, \alpha_s) - {R}(T, C, w, \alpha_s),
    \label{eq:prp_delta_Rc}
\end{equation}
%  $R$ is the net upward radiation flux at tropopause,
where $T$, $C$, $w$ and $\alpha_s$ are temperature, cloud properties (e.g., cloud fraction, cloud water content), water vapor and surface albedo, respectively. $C'$ represents the altered cloud properties in perturbed climate; that is $C'=C+\Delta C$, and so forth for other variables. 
%The cloud radiative effect (CRE) is defined as the difference between total-sky and clear-sky net fluxes at top of the atmosphere (TOA). 
%As we will show below, the CRE method has some limitations. 
In contrast, the change in net CRE at TOA between control and perturbed simulations is defined as
% As the CRE at TOA is defined as the radiation difference between total- and clear-sky conditions, 
\begin{equation}
    \Delta \mathrm{CRE}_{\text{net}} = \underbrace{\left[~{R}(T', C', w', \alpha'_s) - {R}(T', 0, w', \alpha'_s) ~\right]}_\text{\textcolor{black}{Net $\mathrm{CRE}$ in perturbed climate}} - 
    \underbrace{\left[~{R}(T, C, w, \alpha_s)  - {R}(T, 0, w, \alpha_s) ~\right]}_\text{\textcolor{black}{Net $\mathrm{CRE}$ in control climate}},
    %\overline{CRE}_2 - \overline{CRE}_1
    \label{eq:delta_CRE}
\end{equation}
as the CRE at TOA is computed from the radiation flux difference between total- and clear-sky conditions. For the case where there is no cloud feedback $\Delta C=0$, thus $C'=C$ and $\delta_C {R}=0$ in \eqref{eq:prp_delta_Rc}. The change in net CRE in \eqref{eq:delta_CRE} becomes
\begin{equation}
    \begin{aligned}
        \Delta \mathrm{CRE}_{\text{net}} &= \left[~{R}(T', C, w', \alpha'_s)  - {R}(T', 0, w', \alpha'_s) ~\right] - 
        \left[~{R}(T, C, w, \alpha_s) - {R}(T, 0, w, \alpha_s) ~\right] \\
         & = \underbrace{\left[~{R}(T', C, w', \alpha'_s)  - {R}(T, C, w, \alpha_s)  ~\right]}_\text{Change in total-sky flux} - \underbrace{\left[~ {R}(T', 0, w', \alpha'_s) - {R}(T, 0, w, \alpha_s) ~\right].}_\text{Change in clear-sky flux} \\
    \end{aligned}
    \label{eq:delta_CRE_no_cld_fbk}
\end{equation}

% \begin{figure}
%     \centering
%     \includegraphics[width=1\linewidth]{{figs/methods/fig2_Soden2004_PRP_CRF_comparison}.pdf}
%     \caption{Comparison of the (a) shortwave, (b) longwave, and (c) net cloud feedback parameters from the PRP and $\Delta$CRF (or $\Delta$CRE) methods. Adapted from Fig. 2 of \cite{Soden2004}.}
%     \label{fig:cmp_PRP_CRE_Soden}
% \end{figure}

If we assume that $\Delta T$, $\Delta w$, and $\Delta \alpha_s$ are not zero (i.e. $T'\neq T$, $w'\neq w$, and $\alpha_s'\neq \alpha_s$), the only way to get $\Delta \mathrm{CRE}_{\mathrm{net}}=0$ would be for the changes in total-sky flux (term within first bracket in \eqref{eq:delta_CRE_no_cld_fbk}) and clear-sky flux (term within second bracket in \eqref{eq:delta_CRE_no_cld_fbk}) due to non-cloud feedbacks to be equal. However, this is not always the case. An explanation with a simple model is presented in \secref{sec:cld_masking_effect}. %, but here we provide some simulation results from GCMs first.

% Taking the results from \cite{Soden2004} as an example, shortwave (SW), longwave (LW) and net cloud feedback parameters estimated from PRP and $\Delta$CRE mthods from several GFDL AM2 (version numbers denoted as p\#) models are shown in \figref{fig:cmp_PRP_CRE_Soden}. The SW cloud feedback calculated from PRP method is smaller than the ones from $\Delta$CRE method, while the LW cloud feedback from PRP method is larger than that from $\Delta$CRE approach. In total, the magnitude of net cloud feedback from $\Delta$CRE method is underestimated ranges from 0.3 to 0.4 Wm$^{-2}$K$^{-1}$ in AM2 models compared to PRP method. 

% \begin{figure}
%     \centering
%     \includegraphics[width=0.65\linewidth]{{figs/methods/Soden_Held_cld_fbk_and_delta_CRF}.png}
%     \caption{The cloud feedback parameter plotted as a function of the change in global-mean net cloud radiative forcing per degree change in global surface temperature. Adapted from Fig. 4 of \cite{Soden2006}.}
%     \label{fig:cld_fbk_and_CRE_Soden_Held}
% \end{figure}

% \begin{figure}
%     \centering
%     \includegraphics[width=0.9\linewidth]{{figs/methods/cld_fbk_from_kernel_CRF_adjust_forcing}.pdf}
%     \caption{Multimodel ensemble-mean maps of the cloud feedback estimated as (a) the residual of the kernel calculations, (b) the change in cloud forcing, and (c) the change in cloud forcing after adjusting for the effects of cloud masking on noncloud feedbacks and external radiative forcing. Both the cloud feedback and cloud-masking adjustments to the change in cloud forcing are estimated using the GFDL kernel. Adapted from Fig. 11 of \cite{Soden2008}.}
%     %  Only those models for which both the cloud feedback and CRE were available are included in the ensemble mean.
%     \label{fig:cld_fbk_and_CRE_pattern}
% \end{figure}
%Similarly, as shown in \figref{fig:cld_fbk_and_CRE_Soden_Held}, 
\cite{Soden2006} found that all the models from IPCC Fourth Assessment (AR4) have a positive cloud feedback in 21$^{\text{st}}$ century climate change experiments, but roughly half the models show a reduction in net CRE (or CRF; and normalized by surface temperature change) in response to climate change. In their study the cloud feedback is estimated by the radiative kernel method. This apparent discrepancy is possibly due to the influence of noncloud feedbacks on the CRE term \citep{Zhang1994,Soden2004}. Therefore, the change of net CRE is not a reliable measure of cloud feedback, as the signs of them are sometimes different. However, it is noted that the cloud feedback is correlated with the change in net CRE, indicating that intermodel differences in cloud feedback can be estimated by the intermodel differences in the changes of CRE \citep{Soden2006,Bony2006,Vial2013}. %In addition, the spatial patterns of cloud feedback from radiative kernel and $\Delta$CRE approaches are similar, although the magnitude shows large difference (\figsref{fig:cld_fbk_and_CRE_pattern}b and \ref{fig:cld_fbk_and_CRE_pattern}c), implying that the $\Delta$CRE method is still valuable to have a quick look of the cloud feedback pattern.

\subsection{Cloud masking effect}
\label{sec:cld_masking_effect}

A modified simple thought experiment from \cite{Soden2008} is adopted here to explain what the cloud masking effect is and why the two right-hand terms in \eqref{eq:delta_CRE_no_cld_fbk} are usually not equal with each other. As illustrated in \figref{fig:cld_masking_effect}, we assume part of the grid is covered by high clouds (cloud fraction is $f$), and the water vapor contents are $q_1$ and $q_2$ for clear and cloudy subgrid regions, respectively. Here we focus on longwave radiation only and assume the longwave radiation emitted by water vapor is a linear function of its content, that is $\alpha + \beta q$, where $\alpha$ and $\beta$ are assumed linear coefficients and $\beta$ is negative so that the outgoing longwave radiation decreases with the increase of water vapor. The outgoing longwave radiation emitted from clouds is regarded as a constant $W$.

\begin{figure}
    \centering
    \includegraphics[width=0.5\linewidth]{{figs/methods/cloud_masking_effect}.pdf}
    \caption[Illustration of cloud masking effect]{Illustration of cloud masking effect on noncloud feedbacks in a grid box.}
    \label{fig:cld_masking_effect}
\end{figure}

If there is no clouds in this grid (i.e. $f=0$), the net longwave radiation flux at TOA (downward positive) is $-[\alpha + \beta(q_1+q_2)]$. When clouds are present, the grid averaged net longwave flux at TOA becomes
\begin{equation}
    R = - (\alpha + \beta q_1)(1-f)- Wf,
    \label{eq:cld_masking}
\end{equation}
as the longwave raditation emitted from water vapor $q_2$ is masked by clouds. Consider a climate change case in which the water vapor and cloud fraction change a bit, and the change in $R$ can be written as follows by differencing \eqref{eq:cld_masking}:
\begin{equation}
    \delta R = \delta R_f + \delta R_q,
\end{equation}
where
\begin{equation}
    \delta R_f = (\alpha + \beta q_1 - W)~\delta f,
    \label{eq:lw_change_due_to_cld}
\end{equation}
and
\begin{equation}
    \delta R_q = -(1-f)\beta~\delta q_1.
    \label{eq:lw_change_due_to_q}
\end{equation}
Now reconsider the situation in \eqref{eq:delta_CRE_no_cld_fbk}, in which there is no cloud feedback ($\delta f=0$), so the longwave radiation change at TOA due to clouds is also zero, namely $\delta R_f=0$ in \eqref{eq:lw_change_due_to_cld}. However, as for the longwave radiation flux change due to water vapor perturbation, the cloud fraction $f$ is also included as in \eqref{eq:lw_change_due_to_q}, indicating that clouds have masking effect on the water vapor feedback and the $\delta R_q$ under clear-sky should be different from the one under cloudy conditions. 

This simple thought experiment has illustrated that the presence of clouds can have an impact on the radiation associated with noncloud variables. That is why the two right-hand terms in \eqref{eq:delta_CRE_no_cld_fbk} are usually not equal with each other, and it also implies that estimating cloud feedback from changes of cloud radiative effect is probably not a good option due to the cloud masking effect.

\subsection{Cloud radiative kernel method}
\label{sec:Zelinka_method}

In general, previous methods introduced in \secref{sec:three_climate_fbk_methods} can give us an estimate of integrated quantity of cloud feedback. Of course, methods such as the PRP and radiative kernel can also generate other types of cloud feedbacks by perturbing the corresponding properties, but it is usually hard to do so. To solve this problem, \cite{Zelinka2012computing1,Zelinka2012computing2} proposed a new technique based on cloud radiative kernel \index{Radiative kernel!cloud} and the histograms of cloud fraction partitioned by CTP and $\tau$, which can easily attribute the contributions of specific types of cloud changes to cloud feedback.

\begin{figure}[ht]
    \centering
    \includegraphics[width=0.8\linewidth]{{figs/methods/cld_kernel_Zelinka}.pdf}
    \caption[Global and annual mean of Zelinka cloud radiative kernels]{Global, annual, and ensemble mean (a) longwave, (b) shortwave and (c) net cloud radiative kernels. Adapted from Fig. 1 of \cite{Zelinka2012computing1}. \copyright American Meteorological Society. Used with permission.}
    % In each model, the kernels have been mapped to the control climate's clear-sky surface albedo distribution before averaging in space; thus, the average kernels are weighted by the actual global distribution of clear-sky surface albedo in each model.
    \label{fig:cld_rad_kernel_Zelinka}
\end{figure}

In this method, the cloud radiative kernel $K$\index{Cloud radiative kernel} in each histogram bin of \figref{fig:cld_rad_kernel_Zelinka} is defined as
\begin{equation}
    K = \frac{\partial R}{\partial C},
\end{equation}
which quantifies the sensitivity of TOA radiative flux to cloud fraction changes ($\Delta C$), and is estimated offline from radiation transfer code for each CTP-$\tau$ bin. The process is complicated and a detailed description can be found from \cite{Zelinka2012computing1}. The longwave, shortwave and net cloud radiative kernel results are shown in \figref{fig:cld_rad_kernel_Zelinka}, and the major features are:
\begin{enumerate}
    \item The longwave cloud radiative kernel is positive for all bins, as the longwave CRE is positive. It increases dramatically with cloud top height, and is small near surface as the cloud top temperature contrast with surface is small.
    \item The shortwave cloud radiative kernel is negative for all bins, as the shortwave CRE is negative. Its magnitude increases dramatically with optical depth, and almost insensitive to cloud top height.
    \item The net cloud radiative kernels for the lower and thicker clouds are negative, as the shortwave reflection exceeds longwave trapping; while for higher and thinner clouds, the net cloud radiative kernel are positive, as longwave trapping exceeds shortwave reflection.
\end{enumerate}
Multiplying the cloud radiative kernel $K$ by the change in cloud fraction histogram ($\Delta C$, differencing the histogram outputs from cloud simulator), one can estimate the contribution of each cloud type to the change in TOA radiation associated with climate change:
\begin{equation}
    \Delta R = K \Delta C
\end{equation}
and hence the cloud feedback is calculated as follows:
\begin{equation}
    \lambda_c = K \frac{\Delta C}{\Delta T_s} = \frac{\Delta R}{\Delta {T}_s}.
\end{equation}
Note that the $\Delta R$ and cloud feedback parameter $\lambda_c$ are function of CTP, $\tau$, latitude, longitude and time, so it can be used to estimate cloud feedback from certain cloud types (according to CTP and $\tau$). The cloud radiative method, combining with the ISCCP histogram outputs from COSP, will be used in \chapref{ch:cld_fbk}.
\index{Cloud feedback!Calculation method|)}

\section{Low cloud amount proxy}
\label{sec:method_low_cld_amount_proxy}

Several low cloud amount proxies have been proposed in previous studies and used for low cloud amount predictions \citep[e.g.,][]{Kawai2006, Joshi2015, Collins2004, Guo2014, Kawai2019}. Here we list the expressions of these proxies for reference, as they are used in \chapref{ch:simple_cld_scheme}.

\subsubsection{Low-tropospheric stability}
\index{Low-tropospheric stability}

\cite{Klein1993} studied the seasonal cycle of low stratiform clouds with data from the Earth Radiation Budget Experiment and found that the low-tropospheric stability (LTS) \index{LTS} has a good linear relationship with the low stratus amount, which is applied in the Community Atmosphere Model \citep{Collins2004}. In the scheme, LTS can be seen as a measure of the inversion strength and is defined as the potential temperature difference between 700 hPa ($\theta_{700}$) and surface ($\theta_s$),
\begin{equation}
	LTS=\Delta \theta \equiv \theta_{700} - \theta_s.
\end{equation}

\subsubsection{Estimate inversion strength}
\index{Estimate inversion strength}

\begin{figure}[ht]
    \centering
    \includegraphics[width=0.5\linewidth]{{figs/methods/Wood_EIS}.png}
    \caption[Idealized profile of lower-tropospheric structure]{Idealized profile (thick solid line) of lower-tropospheric structure during periods of undisturbed flow. Moist adiabats are shown as light dotted lines. Adapted from \cite{Wood2006}. \copyright American Meteorological Society. Used with permission.}
    \label{fig:wood_eis_atm_profile}
\end{figure}

Despite the wide use of LTS in climate models, \cite{Wood2006} argued that it has yet to be demonstrated whether the observationally derived LTS-CF relationships will hold in a changed climate. They proposed a new proxy called estimate inversion strength (EIS) \index{EIS} to represent the planetary boundary layer inversion strength and proved that it is better to indicate the low stratiform cloud cover. An idealized profile of lower-tropospheric structure is shown by a thick solid line in Figure \ref{fig:wood_eis_atm_profile}, where the atmospheric conditions roughly follow a dry adiabat below the lifting condensation level (LCL), and then a moist adiabat above. The EIS is defined as
\begin{equation}
    \mathrm{EIS}=\mathrm{LTS}-\Gamma_{m}^{700} z_{700}+\Gamma_{m}^{\mathrm{LCL}} z_{\mathrm{LCL}},
    \label{eq:eis}
\end{equation}
or simplified as
\begin{equation}
    \mathrm{EIS}=\mathrm{LTS}-\Gamma_{m}^{850}\left(z_{700}-z_{\mathrm{LCL}}\right),
\end{equation}
where 
\begin{equation}
    %\Gamma_{m}(T, p)=\frac{g}{c_{p}}\left[1-\frac{1+L_{v} q_{s}(T, p) / (R_{a} T)}{1+L_{v}^{2} q_{s}(T, p) / (c_{p} R_{v} T^{2})}\right],
    \Gamma_m(T, p) = \frac{g}{c_{p}}\left(1-\frac{1 + \frac{L_{v} q_{s}(T, p)}{R_{a} T}}{1+ \frac{L_{v}^{2} q_{s}(T, p)}{c_{p} R_{v} T^{2}}}\right),
    \label{eq:gamma_m}
\end{equation}

\begin{equation}
	z_{700}=\frac{R_{a} T_{0}}{g} \ln \left(\frac{p_{0}}{700~ \mathrm{hPa}}\right).
	\label{eq:z700}
\end{equation}
In \Eqref{eq:eis}, $\Gamma_{m}$ is the moist-adiabatic potential temperature gradient, $z_{700}$ is the height of 700 hPa, $z_{\mathrm{LCL}}$ is the height of the lifting condensation level. In \Eqref{eq:gamma_m}, $L_v=2.47\times 10^6$ J kg$^{-1}$ is the latent heat of vaporization. $q_s(T,p)$ is the saturation mixing ratio, and is a function of temperature and pressure derived from the Clausius-Clapeyron equation. $R_a=287.04$ J kg$^{-1}$ K$^{-1}$ and $R_v=461.50$ J kg$^{-1}$ K$^{-1}$ are the gas constants for dry and water vapor, respectively, $g=9.8$ m s$^{-2}$ is the gravitational acceleration, and $c_p=1006$ J kg$^{-1}$ K$^{-1}$ is the specific heat of air at constant pressure. In \Eqref{eq:z700}, $T_0$ and $p_0$ (in units of hPa) are the surface temperature and pressure respectively.

\subsubsection{Estimated cloud-top entrainment index}
\index{Estimated cloud-top entrainment index}

Recently proposed predictors of low cloud amount are also considered in this study: one is the estimated cloud-top entrainment index (ECTEI)\index{ECTEI} by \cite{Kawai2017} and the other is the estimated low-level cloud fraction (ELF) by \cite{Park2019} (see \Eqref{eq:ELF} in \chapref{ch:simple_cld_scheme}). The ECTEI is a modification of estimated inversion strength (EIS) and takes into account a cloud-top entrainment criterion and is defined as
\begin{equation}
    \mathrm{ECTEI} \equiv \mathrm{EIS} - \beta\left(L_v / c_{p}\right)\left(q_{0}-q_{700}\right),
    \label{eq:ectei_Kawai}
\end{equation}
where
\begin{equation}
    \beta=(1-k) C_{\mathrm{q\_gap}}.
\end{equation}
In \Eqref{eq:ectei_Kawai}, $L_v$ is latent heat of vaporization, and $c_p$ is the specific heat of air at constant pressure. The values of both parameters are the same as the ones in  \Eqref{eq:gamma_m}. $q_{0}$ and $q_{700}$ are the specific humidity at surface and 700 hPa respectively. The coefficient $C_{\mathrm{q\_gap}}$ is the ratio of the total water specific humidity ($q_t$) gap at the inversion and the $q$ difference between the surface and 700 hPa. \cite{Kawai2017} estimated the $C_{\mathrm{q\_gap}} =$ 0.76, $k=$ 0.70 and $\beta =$ 0.23 based on radiosonde observation data off the coast of Peru. 

%\cite{Park2019} argued that the use of LTS and EIS as a global proxy for low cloud amount is limited due to their weaker and inconsistent relationship with low cloud amount over land. They kept the calculation of height of inversion, the term with which was neglected in the derivation of EIS \citep{Wood2006}.
