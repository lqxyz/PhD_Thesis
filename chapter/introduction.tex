\chapter{Introduction}
\label{ch:introduction}

\section{The Earth’s radiation budget}
The Earth's climate is driven by the energy flow into and out of the system. The incoming solar radiation (yellow fluxes in \figref{fig:earth_energy_budget}) reaches the Earth at the top of the atmosphere (TOA), then goes through the atmosphere and arrives at the Earth surface. During this process, approximately two thirds of this shortwave (SW) radiation is absorbed by the Earth surface and atmosphere, and roughly one third of this energy is reflected back to space. The the surface and atmosphere are heated by this incoming solar radiation, and they also re-emit the longwave (LW) radiation (purple fluxes in \figref{fig:earth_energy_budget}) to keep a relatively stable temperature. Globally, the annual mean incoming solar radiation flux is about 340 Wm$^{-2}$, the reflected solar radiation flux is around 100 Wm$^{-2}$ and the outgoing longwave radiation (OLR) is close to 240 Wm$^{-2}$ at the TOA for period 2000--2010 \citep{Stephens2012update}. These three components balance with each other, with a small positive imbalance (about 0.6 Wm$^{-2}$) at the TOA. A more recent estimate from \cite{Wild2015} (see their Fig. 1) indicates that the gloabl mean OLR is about 239 Wm$^{-2}$, and the TOA energy imbalance is about 1 Wm$^{-2}$.
%But due to the accuracy of the observation itself, it is hard to track the causes of these imbalance.

\begin{figure}[ht]
	\centering
	\includegraphics[width=1\linewidth]{{figs/literature_review/earth_enery_budget_Stephens2012}.png}
	\caption[The global annual mean energy budget of Earth for the approximate period 2000–2010 from \cite{Stephens2012update}]{The global annual mean energy budget of Earth for the approximate period 2000–2010. All fluxes are in Wm$^{-2}$. Solar fluxes are in yellow and infrared fluxes in purple. The four flux quantities in purple-shaded boxes represent the principal components of the atmospheric energy balance. Adapted from \cite{Stephens2012update}.}
	\label{fig:earth_energy_budget}
\end{figure}

As shown in \figref{fig:earth_energy_budget}, the energy budget at the Earth's surface is more complicated than at the TOA. When the incoming solar radiation reaches the surface, the majority (about 165 Wm$^{-2}$) of this solar radiation is absorbed by the surface and only a small portion (about 23 Wm$^{-2}$) is reflected back to space. Of course, these are global mean results and it would be different over certain areas such as Arctic where the surface albedo is large. The global annual mean LW radiation emitted from the surface is about 398 Wm$^{-2}$. Much of this is absorbed by the atmosphere (such as the greenhouse gases, aerosols and clouds), and only a small part (about 20 Wm$^{-2}$) can pass through the atmospheric window region (a portion of the infrared spectrum where there is almost no atmospheric absorption) reaching the TOA directly. The atmosphere can re-emit the absorbed LW radiation both upward and downward, and downward part (about 346 Wm$^{-2}$) can reheat the surface. In addition, due to the temperature and moisture difference between the surface and atmosphere, the surface is also cooled by the latent heat flux (about 88 Wm$^{-2}$) and sensible heat flux (about 24 Wm$^{-2}$) through the turbulent movement of atmosphere. In total, the surface energy budget is balanced by the downward/upward SW and LW radiation, the sensible and latent heat fluxes, but it has much larger uncertainty than at the TOA.

\section{Climate feedback}

\section{Clouds and cloud radiative effect}

Clouds usually cover more than half areas of the Earth at any given time \citep{Houze2014,Ramanathan1989}, which is supported by recent International Satellite Cloud Climatology Project (ISCCP)\index{ISCCP} H-series products \citep{Young2018}, as shown in \tabref{tab:statistics_cld_amt}. Although they vary among different regions, the cloud amounts are usually larger over ocean regions than over lands, as the ocean provides a abundance of water vapor.

\begin{table}[htp]
\centering
%\scriptsize
\caption{Statistics of the annually averaged total cloud amount (\%) for various regions, which are from the ISCCP H-series data sets \citep{Young2018} from 1984 to 2014.}
\vspace{0.5em}
\begin{tabular}{cccc}
	\toprule
	Region & Ocean & Land &  Total\\
	\midrule
	Global & 71.7 & 54.8 & 66.1 \\
	15$^\circ$S--15$^\circ$N&  62.4&  63.5& 62.6 \\
	15$^\circ$N--35$^\circ$N&  60.1&  46.6& 55.2\\
	15$^\circ$S--35$^\circ$S&  65.0&  48.3& 61.4\\
	35$^\circ$N--60$^\circ$N&  80.9&  64.6& 72.5 \\
	35$^\circ$S--60$^\circ$S&  84.0&  65.0& 83.5 \\
	60$^\circ$N--90$^\circ$N&  68.9&  62.0& 66.5\\
	60$^\circ$S--90$^\circ$S&  80.1&  44.3& 60.1 \\
	\bottomrule
\end{tabular}
\label{tab:statistics_cld_amt}
\end{table}

\begin{figure}[ht]
	\centering
	\includegraphics[width=1\linewidth]{{figs/literature_review/spatial_pattern_of_CRE_from_IPCC_ch7}.png}
	\caption{Distribution of annual-mean top of the atmosphere (a) shortwave, (b) longwave, (c) net cloud radiative effects averaged over the period 2001–2011 from the Clouds and the Earth’s Radiant Energy System (CERES) Energy Balanced and Filled (EBAF) Ed2.6r data set. Adapted from Fig. 7.7 of The Fifth Assessment Report (AR5) of the Intergovernmental Panel on Climate Change (IPCC) \citep{Stocker2013}.}
	\label{fig:CRE_from_IPCC}
\end{figure}

\section{Cloud feedback}

\section{Cloud scheme}
\label{sec:cld_scheme_history}
\index{cloud scheme}

\subsection{Overview}
\begin{figure}[ht]
	\centering
	\includegraphics[width=0.8\linewidth]{{figs/literature_review/Schematic_partial_cloud_cover_in_gridbox}.pdf}
	\caption{Schematic showing that partial cloud cover in a grid box (1D) when temperature or humidity fluctuations exist. The blue line shows humidity and the red dashed line indicates saturation mixing ratio of the grid box. The shaded regions are cloudy parts as the humidity exceeds the saturation mixing ratio.}
	\label{fig:schematic_partial_clouds}
\end{figure}

At present the typical horizontal resolution of the GCMs is 50-200km, but the clouds usually involve the air motions in mesoscale and convective scale \citep{Houze2014}, which are usually in sub-grid scale both horizontally and vertically, implying that cloud processes are hard to be explicitly resolved by the GCMs. In this case, the ``parameterization" becomes a practical way to build cloud schemes. The parameterization is to represent the effects of the smaller-scale processes (turbulence, cloud microphysics, convection, etc.) in terms of the large-scale states (such as velocity, temperature, pressure, humidity) \citep{Randall2003}, which could be seen as a way to find potential relationships between the unknown and known variables \citep{Randall1989}.


\subsection{Relative humidity schemes}
\index{cloud scheme!RH scheme}
Previous studies have investigated various ways to represent clouds in climate models. For example, \cite{Holloway1971} prescribed the clouds externally with climatological data without dynamic interplay with the other components of the model. Some early modelling studies made the assumption that a grid box in the model is either fully saturated or totally unsaturated. However, this assumption is not reasonable enough as the humidity can distribute unevenly within a grid box, suggesting that condensation can occur even the relative humidity is less than 100\%. A general idea is to link the cloud cover with the relative humidity, as one can expect that the amount of condensation would increase with the increase of mean humidity of the model grid box, which is the basis for some diagnostic methods.

Diagnostic schemes predict the cloudiness based on the model variables empirically or statistically. In these schemes, the clouds can be linked to atmospheric outputs such as relative humidity, vertical velocity and static stability, among which the linear relationship between cloud fraction against RH could be simplest one. For example, \cite{Smagorinsky1960} found empirically that non-convective cloud amount correlated with the average relative humidity in the respective layers, arguing that the non-precipitating condensation depend only on the accumulated history of vertical motion, which can be reflected by the humidity. \cite{Ricketts1973} obtained roughly linear relationship between cloud amount and observed relative humidity but commented that the relationship is somewhat indefinite. %(\figref{fig:Smagorinsky_RH_cld})

Water vapor generally distributes heterogeneously in the grid box, so the averaged RH within a box should be less than 1 for a partial coverage of clouds. Previous studies usually adopt the critical relative humidity $RH_{crit}$ as the minimum threshold for clouds to form, which is often left as a free parameter that can be tuned during model development (e.g. \citealp{Hourdin2017,Kay2012,Mauritsen2012}). For example, \cite{Sundqvist1978} and \cite{Sundqvist1989} find that cloud fraction can be rewritten as a function of critical RH by assuming the water vapour is uniform distributed within the grid box. In general, $RH_{crit}$ decreases with height, but will vary according to different types of clouds. Although the $RH_{crit}$ doesn't have clear physical meaning, it can be used to modify the cloud amounts in different locations. For example, one can increase $RH_{crit}$ asymptotically to nearly unity to prevent the unrealistic circus clouds \citep{Sundqvist1989}. 

As a unique predictor, RH is very simple and useful to diagnose the cloudiness, and it is still widely used in GCMs \citep[e.g.,][]{Gordon1992,Park2014,Pope2000}. However, it is not valid for all the cases. As we can see, some studies also made use of other variables to diagnose the cloudiness. For instances, \cite{Xu1996} developed a semi-empirical scheme to determine the stratiform cloud fraction based on grid-averaged mixing ratio of condensate (cloud water and cloud ice) and RH. As for the scheme provided by \cite{Slingo1987}, both the RH and vertical velocity were taken into account, in which different empirical relations were used for different clouds including low, middle, high and convective clouds.

In summary, the methods based on relative humidity and other predictors are useful to diagnose the cloudiness, which ensures that the clouds can form before the grid box get saturated. One problem for the diagnostic methods is that in most cases the cloud condensate has to be diagnosed or prognosed via other methods \citep[e.g.,][]{Zhang2003, Park2014}, which could lead to some inconsistencies between cloud fraction and cloud condensate (e.g. \citealp{Gregory2002, Tompkins2005}).

% \begin{figure}
% 	\vspace{-0.3cm}
% 	\centering
% 	\includegraphics[width=0.6\linewidth]{{figs/literature_review/Smagorinsky1960}.png}
% 	\caption{Empirically determined relation of mean relative humidity $h$ in the layers 1000-800mb, 800-550mb and 550-300mb with cloud amount $c$ classed as low, middle and high clouds. Adapted from Figure 1 of \cite{Smagorinsky1960}.}
% 	\label{fig:Smagorinsky_RH_cld}
% \end{figure}


\subsection{Statistical schemes}
\label{sec:PDF_cld_scheme}
\index{cloud scheme!statistical or PDF scheme}

In contrast, the prognostic approach \citep[e.g.,][]{Tiedtke1993} is to explicitly calculate the clouds related variables, such as cloud water content, in order to pursuing a unification of all clouds processes, which is more realistic in some degree and requires more physical basis and interactions with other parts of the models. Another widely used cloud prediction method is statistical scheme, in which the cloud fraction and in-cloud liquid water/ice are determined based on the assumed probability distributions of subgrid variability of thermodynamic properties. As the cloud related variables such as moisture and temperature are not the same everywhere but distributed randomly within the grid box, it is natural to assume that the cloud cover depends on the distribution of moisture, sometimes on the joint distribution of moisture and temperature. As shown in a very early work, \cite{Sommeria1977} gave up the assumption that a grid is either entirely saturated or unsaturated in the climate models and proposed the idea to use the statistical distribution of moisture within the grid box. For example, given the probability distribution function (PDF) of the total water ($q_t$ is the mixing ratio) in grid box, the cloud fraction ($CF$) can be calculated as 
\begin{equation}
    CF=\int_{q_s}^{\infty}\text{PDF}(q_t)\operatorname{d}q_t,
\end{equation}
and the cloud water content ($q_c$ is the mixing ratio of cloud water) is
\begin{equation}
    q_c=\int_{q_s}^{\infty}(q_t-q_s)\text{PDF}(q_t)\operatorname{d}q_t,
\end{equation}
where $q_s$ is the saturation mixing ratio in both formulations.

However, the shapes of subgrid-scale PDF of total water specific
humidity, saturation deficit, or a combined variable of liquid water and potential temperature are difficult to determine due to limit of observational data, so sometimes the model data are also used \citep{Bony2001}. Additionally, many different forms PDF have been proposed in the previous studies. For example, \cite{LeTreut1991} made use of the uniform distribution of total water in the grid box to calculate the clouds cover and liquid water content. Other symmetrical distributions, such as Gaussian distribution \citep{Sommeria1977}, triangular distribution \citep{Smith1990} and skewed distributions, such as lognormal distribution \citep{Bony2001} and beta distribution \citep{Tompkins2002}, have also been employed in numerical models. However, there are also some problems in the distributions. For example, the Gaussian distribution is unbounded, indicating that the maximum cloud condensate mixing ratio might approach infinity, and cloud cover is always large than zero \citep{Tompkins2002}. In general, complicated forms of the PDF need more parameters to fit. But due to the limitation of the data, it is possibly hard to validate the distributions. Linking the statistical cloud scheme to other physical processes seems a promising way to improve cloud simulations. For example, \cite{Qin2018} developed a Gaussian PDF cloud scheme with the PDF variance diagnosed from the turbulent
and shallow convective processes, which could improve the simulation of low marine clouds and alleviate double Intertropical Convergence Zone (ITCZ) problem \citep{Qin2018alleviated}.

The statistical cloud schemes may have better performance than the diagnostic ones, but considering the fact that there is no clouds scheme in Isca currently, it would be useful to implement the simple diagnostic schemes in Isca first, which can be seen as the first step to implemented a hierarchy of cloud schemes.

\subsection{Relationship between relative humidity and statistical schemes}

As discussed in previous section, one has to determine the expression of sub-grid variance (i.e. second order moment) or other higher-order moments in the statistical schemes. In doing so there are two general practices in current studies. The simple case is to use the time-invariant variance \citep[e.g.,][]{Sundqvist1978,Smith1990}, and the other approach is to employ time varying variance, which are usually obtained from other physical processes such as boundary layer scheme or shallow convection schemes \citep[e.g.,][]{Qin2018}. The second case usually provides a more realistic link between clouds and other physical processes \citep{Tompkins2002}. 

Note that there is no distinction between the RH schemes and statistical schemes, although they seem different in forms. As a matter of fact, if the subgrid variance in a statistical scheme is assumed to be time-invariant, it can be reduced to a RH scheme \citep{Tompkins2002,Tompkins2005}. The key is to link the variance with the critical relative humidity. That is to say, critical RH value can reflect the level of sub-grid variance in RH schemes \citep{Quaas2012}. Larger critical RH value means the lower subgrid variability and vice versa. For example, the \cite{Sundqvist1978} RH scheme can be derived by assuming a uniform distribution of total water mixing ratio within a grid box, in which the variance is assumed a constant fraction of the saturation water vapor mixing ratio, and this constant is associated with critical RH value (see Appendix A of \cite{Quaas2012} for a full derivation). Another example is the triangular distribution used by \cite{Smith1990} and \cite{Park2014}, they also obtain the equivalent RH formulation by assuming the variance is related to critical RH. As pointed by \cite{Tompkins2002}, the parameterizations such as \cite{Xu1996}, in which cloud fraction is related to RH and cloud condensate, can be viewed as manifestations of a statistical scheme although where the actual PDF of total water is not known. %, but the time-mean statistics of its integral are.

\section{Research questions and thesis outline}
\label{sec:thesis_layout}

The thesis is arranged as follows:

\chapref{ch:introduction} introduces the backgrounds, motivation, aims and outline of this study. The background parts include the basic ideas of Earth's radiation budget, climate feedback and cloud radiative effect and its feedback.

\chapref{ch:methods} describes the data and methods used in the thesis. The data sets include the satellite observations about clouds and radiation, and other climate variables, such as temperature, relative humidity and vertical velocity, from the reanalysis. The methodology part first gives a brief introduction of Isca model, then presents the methods used in the thesis to calculate climate feedback. This should have two parts: First is about how to employ kernel method in Isca to estimate Planck, lapse rate, water vapor feedbacks for gray and RRTM radiation schemes; Second is to compare the methods to calculate the cloud feedback. Finally, several low cloud proxies used in \chapref{ch:simple_cld_scheme} is listed for reference.

\chapref{ch:polaramplification} mainly presents the results from polar amplification of temperature change in varying albedo simulations. In this chapter, the contributions from different climate feedbacks, forcing and heat transport are quantified through the decomposition method. The major conclusion is that the local lapse rate feedback and Planck feedback (plus heat transport) play important role in determining the warming structure in polar region.

\chapref{ch:simple_cld_scheme} focuses on the parameterization of simple cloud scheme and the evaluation of its simulation of cloud climatology. These include the parameterization of the cloud fraction and cloud optical properties, simulation setup and the comparison of the simulated cloud fraction, radiative flux and cloud radiative effect with observations and CMIP5 models. The comparison consists of the spatial pattern, zonal mean structure and seasonal cycle. %Basically, this chapter is modified from the GMD manuscript. The content to be added is the sensitivity of the scheme to horizontal and vertical resolutions.

The topic of \chapref{ch:cld_fbk} is cloud feedback. First, the cloud feedback simulated from the simple cloud scheme is evaluated. The spatial patterns of longwave, shortwave and net cloud feedbacks, as well as their components such as cloud amount, altitude and optical depth feedbacks, will be investigated and compared with the CMIP models. The possible reasons for the feedback features are to be explored. Then the results from perturbed parameter ensemble (PPE) will be analysed. The cloud feedback spread from the PPE will be checked to see if they could `reproduce' such spread in CMIP models. The causes to the cloud feedback spread in the PPE will be investigated. Base on these results, which parameter or process is more sensitive would be analysed. Finally, the implications for equilibrium climate sensitivity will be discussed.

\chapref{ch:conclusion} summarises the major contents and conclusions of the thesis, and discusses the possible future work.
